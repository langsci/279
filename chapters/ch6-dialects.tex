\documentclass[output=paper]{langscibook} 
\ChapterDOI{10.5281/zenodo.8269240}

% \lsConditionalSetupForPaper{} % Please use this instead of loading localpackages.tex, localbibliography.bib, etc. manually
 
\title{Dialect areas and contact dialectology}  

\author{Péter Jeszenszky\affiliation{University of Bern} and  Anja Hasse\affiliation{University of Zurich} and  Philipp Stöckle\affiliation{Austrian Academy of Sciences, Vienna}}

\abstract{Spatial variation of language has been researched qualitatively and quantitatively for at least 150 years by different sub-disciplines of linguistics, each defining differently what dialects and dialect areas are. Linguists agree, however, that the concept of dialect is vague and the extent of a dialect is fuzzy.  With contact being a crucial driver of linguistic change at sub-language levels, we attempt to sketch the perspective that contact dialectology and related sub-disciplines can offer on this fuzziness with regard to the spatial variation of dialects and dialect areas.
Thus we address contact processes and patterns characterizing individuals, groups, communities, areas and beyond, 
at temporal scales spanning from mundane contact through generations to deeper time enough for dialects to diverge and disappear.
}


\begin{document}
\maketitle
\label{chap_dialects}

\section{Introduction} 
\label{Section:Introduction}

In this chapter, we aim to position research on \textit{dialect areas} within the larger field of language contact studies, and we do so at several scales of space and time, corresponding to the fuzzy extents of dialects and their areas. We cover approaches to dialectology that focus on spatial variation, we review processes leading to the emergent patterns of spatial variation, and we detail factors that drive the processes, without claiming to be comprehensive.
Several sub-fields of linguistics investigate language varieties pertaining to different social groups, often independent of spatial context. In this chapter we do not discuss sociolects\footnote{In our conception, the term dialect is not used for varieties defined by factors other than spatial ones, such as social dialects \parencite[cf.][7--9]{Chambers2004}.}
\textit{per se}, but we do consider the social causes of dialect formation.

Before all else, we review the key terms and concepts that this chapter focuses on. First, we introduce the notion of \textit{dialect area} as the main topic and specify the aims of the chapter. Second, we further explore the terms \textit{dialect} and \textit{dialect continuum}, and third, we establish that notions of dialect contact need to be investigated in dialectology, as they represent the processes at work within and across dialect areas.

\subsection{Dialect areas, dialects, and dialect continua}
\label{Intro:Dialects_and_dialect_areas}
Dialect areas, dialect continua, and dialects are spatially bound sub-systems of languages. Therefore, when examining them, the geographical component is crucial. The spatial structure of linguistic variation follows the first law of geography: “Everything is related to everything else, but near things are more related than distant things” \parencite[][236]{Tobler1970}, reformulated by \textcite[][154]{Nerbonne2007c}, who also specified \textit{nearness} to be spatial: “Geographically proximate varieties tend to be more similar than distant ones.” 

\textit{Dialect areas} are geographical entities in which, essentially, dialects of a language are spoken. Due to the conceptual vagueness of \textit{dialect}, as we will see later, it is equally difficult to define \textit{dialect areas} \parencite[514--517]{Stoeckle2014}. All dialects within a dialect area share a number of linguistic features. Whether all these dialects are still mutually intelligible depends on the spatial granularity taken when defining a dialect area. In a very locally defined dialect area, the chance of mutual intelligibility is higher than in a spatially more extensive dialect area comprising more local dialects. The definition of dialect areas in linguistics differs between various sub-fields and their approaches which range from detecting \textit{isoglosses} to aggregating dialectometrical data, or analyzing speakers’ perception of areas and their linguistic features.

Dialect areas can be defined at different spatial granularities \parencite[e.g.,][]{Montgomery2013}, forming an embedded, hierarchical system similar to landscape names, which, in turn, often lend their names to dialect areas. For example, the Lötschental is a very local dialect area comprising the closely related varieties of a few villages in a remote valley in the Alps. This dialect area lies within the larger dialect area of Walliser German, spoken in a larger system of valleys which are part of the canton of Valais in Switzerland. This dialect area is, in turn, part of the Highest Alemannic dialect area, located in the Swiss cantons of Fribourg and Valais, parts of the canton of Berne and other alpine areas.

In numerous human phenomena, it is the density of contact within and across areas that shapes the spatial variation  \parencite[cf.][]{Hagerstrand1952}. The same holds true for dialects and dialect areas: the emergent spatial patterns present in the investigation of dialects also point towards the crucial role of contact. 
While these patterns can be observed at the collective level, they result from long-term local interactions between individuals \parencite[cf.][]{Beckner2009}. Communication potential and self-identification are both important factors for the processes as well as outcomes of dialect contact, and they are directly reflected in the spatial distribution of linguistic features. Spatial proximity, in turn, is essential for intensive communication and for the formation of common identity. Besides, social networks and a myriad of factors underlie linguistic interactions that form the dialect.

Given that linguistic patterns are not preordained but emergent, resulting from human social interactions at several levels, the spatial patterns we find in the distribution of dialectal variation reflect dynamic changes in the use of varieties (detailed in Section \ref{Subsection:Processes}), ontogenetic developments in child language acquisition, diachronic changes, political, societal and cultural factors, among others.

If dialect areas comprise dialects, we need to clarify what is a dialect. Books could be filled with attempts to define a dialect and to distinguish dialects from languages. The boundary between a dialect and a language can be fuzzy, and varieties can change their perceived or officially recognized status from dialect to language and vice versa \parencite[12]{Auer1996}. In this chapter, we cannot offer a complete overview of this terminological issue, but we need to contextualize our notion of \textit{dialect}. We do not regard \textit{every} variety of a language as a dialect, as is often done in the Anglo-American tradition. Instead, we follow the continental European tradition: we define a dialect as a variety spoken at a geographically defined place \parencite[230]{Berruto2010}. Remaining in the European tradition,  dialects are often characterized as a linguistic variety that has not undergone standardization. It is mainly a spoken variety (mostly in informal situations) with a non-standardized textual form which might show a wide variation when written.

As mentioned, the spatial extent of such a geographically bound variety can be variable. The spatial granularity at which researchers define dialects corresponds to the structural similarity and/or phylogenetical relatedness of the varieties. Revisiting the example above, if we consider all Highest Alemannic varieties to be one dialect, we would see a lower number of shared features across this particular linguistic system than we might find if we consider the varieties spoken in the valley of Lötschental to form one dialect.\footnote{The problem of geographical demarcation which becomes apparent in this example is discussed in detail by \textcite[1--8]{Lameli2013} under the term \textit{areal-typological complexity} (Germ.: \textit{arealtypologische Komplexität}).}

Dialects often form a \textit{continuum} in which differences between neighboring dialects might be small, while distant dialects might no longer be mutually intelligible \parencite[cf.][5--7]{Chambers2004}.\footnote{Perceptual dialectology emphasizes the role of perceived differences of dialect features as well as perceived dialect boundaries \parencite[]{Niedzielski2000, Cramer2016}.} Such a continuum features boundaries between individual dialect areas that are fuzzy, often similar to a gradual transition \parencite[][5]{Pickl2016}. One such example is the West Germanic varieties spoken between the coast of the North Sea and the Alps \parencite[5--7]{Chambers2004}.\footnote{In this particular example, contact occurs across dialects, often termed horizontal contact, and between dialects and various standard varieties, such as Standard German, often termed vertical contact \parencite{Auer2011}.} The differences between dialects are cumulative, as \textcite[51]{Bloomfield1933} already noted: “The difference from place to place is small, but, as one travels in any one direction, the differences accumulate, until speakers, say from opposite ends of the country, cannot understand each other, although there is no sharp line of linguistic demarcation between the places where they live." These degrees of cumulative differentiation are often the basis for distinguishing dialect areas within a dialect continuum in a quantitative manner, e.g., in dialectometry. 

\subsection{Dialect areas and contact linguistics}

Dialect contact, that is, communication between speakers of different dialects of a specific language, has a crucial role in shaping areal patterns found in dialect continua  because it may cause diffusion of features within and across dialect areas. However, the effects of a high number of shared grammatical features and phylogenetic relatedness, both given between neighboring dialects, are not clear.\footnote{For a discussion about the role of structural compatibility vs. phylogenetic relatedness, see \citet[chapter 2]{Thomason1988}.} For instance, structural similarity might take on the role of the facilitator of contact but it can also be the result of contact \parencite[417--420]{Bowern2013a}. In turn, the intensity of contact depends on the potential for contact, which is higher between certain dialects due to spatial proximity (\citeauthor{Thomason1988} \citeyear*[50]{Thomason1988};  \citeauthor{Matras.2007} \citeyear*[31]{Matras.2007}). This spatial proximity is, of course, also a function of speakers' movements. With people becoming more mobile, their contact opportunities also widen, opening new horizons for dialect contact and change.

In dialectology, contact is rather understudied. Traditional dialectology addresses questions about the spatial distribution of linguistic features in dialect areas and, thereby, the characteristics and differences of various dialects. Contact linguistics, especially areal linguistics, on the other hand, often focuses on typologically distant and/or phylogenetically non-related languages (for more, see Chapter \ref{chap_areas}).

Furthermore, when dealing with contact, sociolinguists and dialectologists have typically assumed that contact leads to simplification, while typologists have described the outcomes of contact-induced change as complexification in some cases \parencite[13--23]{Trudgill.2011}.

In the following sections, we reflect on contact scenarios that are relevant for dialect change, contributing to the spatial variation in dialects. Afterwards, we review the factors  essential for understanding dialect contact.

\section{Approaches} 
\label{Section:Approaches}

 In this section, we first discuss different approaches within dialectology that focus on geography, that is, the spatial distribution of dialectal features and different ways of detecting and defining dialect areas based on these features. Second, we  present approaches that focus on the social aspects of dialect variation and contact, in the spirit of Anglo-American sociolinguistics. 

\subsection{Geography in focus}
\label{Approaches: geography in focus}
In traditional dialectological models (as well as in many modern approaches), geographic space is generally regarded as a physical container in which dialects and languages are situated. According to this approach, the possibility for speakers to move within space, and, thus to be in contact with others, is largely determined by geographic conditions. 

Usually, the objects of study are \textit{base dialects},  commonly understood as “the most ancient, rural, conservative dialects" \parencite[7--8]{Auer2005}. Thinking in terms of base dialects often means that local speech communities are viewed as homogeneous systems, following the  “one place --- one variety” principle \parencite[196]{Stoeckle2016a}.
Typical research outcomes were meticulous descriptions of the dialects of single locations in the neogrammarian tradition (in German dialectology called \textit{Ortsgrammatik} or \textit{Ortsmonographie}).\footnote{\citet{Winteler1876} authored one of the first books of this kind, setting a model for many other monographs. For an overview of dialect descriptions in the neogrammarian tradition,  see \citet{Murray2010}.} This meant, however, a lack of a more holistic view of dialect areas. The compilation of dialect atlases addressed this problem, although maps could only represent individual features rather than grammatical systems and their interaction. 

Traditional dialectology aimed to record and document the most archaic and most typical dialectal forms still viable at a location, usually based on information provided by so-called NORMs  (non-mobile, older, rural, male speakers) \parencite[29]{Chambers2004}. Despite thoroughly documenting the base dialect, descriptions of individual dialects and dialect atlases hardly considered intra- and inter-speaker variation, atlases visualized variation across locations. Subsequently, contact was only considered between locations and not within communities. In these approaches, dialects were conceptualized as homogeneous systems with villages as their loci. Potential contact between these discrete systems could be traced by single  features shared between two or more places (although  these similarities can be due to shared ancestors in a dialect area). 

Communication potential, shaped by the possibility of transportation and its routes, was seen as a precondition for contact. Therefore, topographic factors, such as mountain ranges or rivers, were always given high relevance \parencites[see e.g.][40--41]{Paul1880}[29--33]{Pickl2014}, in addition to artificial constructs such as political (especially national) borders, former administrative areas, or older inter-tribal borders \parencites[cf.][chap.~2]{Haag1898}[]{Derungs2019}

The spatial distribution of linguistic features was often visualized using boundaries, in the case of individual phenomena,  isoglosses. They were usually represented as sharp lines on maps, delineating the areas where a certain variant corresponding to a feature is assumed to dominate. In practice, drawing isoglosses usually entails ignoring sites that would render the resulting areas less homogeneous by corresponding to the opposing variant, and thus being located on the ``wrong side" of the theoretical line. Technically, these sites are considered outliers. It is understood that such ``smoothing" is in the interest of a meaningful interpretation, needed to identify underlying regional linguistic patterns, and is employed by almost all research in dialectology \parencite[82]{Grieve2014}. Importantly, however, isoglosses leave room for misinterpretations about the possible gradual nature of the transition between the usage areas of the abutting variants. \textcite[][5]{Francis1983} seeks confirmation for the sentiment in the linguistic community that such boundaries do “not mark a sharp switch from one word to the other, but the center of a transitional area where one comes to be somewhat favored over the other.”

While isoglosses present the geographic distributions of individual phenomena, a central goal in dialectology is to construct dialect areas based on multiple features. In these regards, \textcite[191--198]{Lameli2019} distinguishes between \textit{evaluating} and \textit{quantifying} approaches. 

In evaluating approaches, dialect data is interpreted, and variants are selected according to their typicality or their significance in the language system. In some cases, individual, mostly phonological phenomena are considered so important with respect to the structure of a variety that they are used as references for classifications. An example of this would be the division of German dialects into Low, Central and Upper German based on the High German consonant shift (such as \textit{maken} > \textit{machen} `to do', \textit{dat} > \textit{das} `the, that', \textit{Appel} > \textit{Apfel} `apple', from north to south). In other cases, variants are combined and generalized, often resulting in dialectal core areas and transition zones, as can be seen for example in Wiesinger's \parencite*{Wiesinger1983} famous classification of German dialects. 

In quantifying approaches, the areal nature of variation in dialects is investigated through the aggregation of numerous dialect features. In this field of research, which dates back to \textcite{Seguy1971} and became popular through the work and different approaches of \textcite[e.g.,][]{Goebl1982} and \textcite[e.g.,][]{Nerbonne2009DataDriven}, similarity  between locations is computed with respect to a large number of features. Although various methods and computational algorithms have been used to determine similarity, the general outcome of these approaches is groupings of locations. The spatial patterns in these groupings can then be interpreted along the approaches of \textit{dialect areas} and \textit{dialect continua} \parencite[cf.][]{Heeringa2001}. The dialect area approach advocates the presence of clear-cut boundaries, and the dialect continuum approach assumes gradual transitions. Similar to dialect variation becoming apparent when aggregating several linguistic features, one can view gradual dialectal variation also at the level of individual linguistic variables
\parencite[cf.][]{Pickl2013b}.\footnote{\citet{Pickl2013b} also provides a historical overview of the topic.} 

Even if the usage of isoglosses means a spotlight on boundaries, most quantitative studies on spatial variation in linguistics focus on the internal homogeneity of their groupings and do not explicitly assess the strength of boundaries between them. As \textcite[][664]{Haas2010} points out, “the linguistic coherence of a region is more important than its boundary.” Although \textit{transition zones} are often described and used in dialectology 
\parencite[e.g.][]{Pickl2013,Scholz2016},
the concept itself lacks a clear definition and interdialectal transitions themselves have rarely been investigated quantitatively or placed along a gradual scale \parencite{JeszenszkyPaper2}.

Recognizing continua and transition zones in data directly leads to the intuitive interpretation of the ``snapshots" of dialectal landscapes, provided by surveys, as clues about possible ongoing changes. 
The description of areal patterns in dialects is a difficult task not only because of the continuous change in language but also because of the elusive nature of data that can be obtained. Even the most prudent data collections might not be representative of the whole population of interest, and, in turn, the population of interest varies across studies. In addition, the amount of data collected might be too small and it might be biased towards a certain subset of the population (e.g., NORMs, tech-savvy, extroverts, depending on the study), therefore a comparative data analysis across surveys is not always possible. 

\subsection{Speaker/hearer in focus}

While the approaches discussed so far deal with the spatial aspects of dialects and therefore treat dialect contact as contact between varieties related to geographic locations (such as municipalities or regions), approaches influenced by sociolinguistics focus on the speakers and their behaviour. Although the basic idea of dialects as spatially defined varieties is still valid, researchers have come to view dialectal variation as a more complex phenomenon which manifests itself not only between, but also within communities and within a single speaker, influenced by socio-demographic factors (such as age, profession, education, mobility etc.) \parencite[cf.][]{Chambers2002}. Consequently, dialect contact is no longer seen as contact between base dialects, but between all types of varieties. 
Moreover, in most modern societies, substantial contact-induced changes usually do not emerge primarily between dialects, but between dialects and an overarching standard variety.\footnote{\textcite{Auer2005} delivers an overview of different dialect/standard constellations in Europe.}  

The traditional focus on base dialects allows for studying contact and change on the horizontal level, that is, between different locations. The sociolinguistic paradigm following Labov’s work \parencite[e.g.,][]{Labov1966} brought a new, vertical dimension to this research. This paradigm focuses on the variation and contact across social strata, that is, the variation along the dialect-standard axis or between lower and higher prestige varieties. Some of the main benefits of including socio-demographic factors into dialectology were new possibilities for the “study of [language] change in progress” \parencite[312]{Bailey2002}. Following the traditional paradigm, it is extremely difficult for dialectal surveys to investigate the same features decades apart. 
There are examples of studies comparing results of surveys from different time periods such as \textcite{Schwarz2015} or \textcite{Streck2012} who contrast maps from the \textit{Sprachatlas des Deutschen Reichs} \parencite{Wenker1888} with material from the \textit{Südwestdeutscher Sprachatlas} \parencite{Steger1989} which was collected about ninety years later. However, these examples are rare and therefore, dialectology has limited ``ground truth" with good spatial granularity regarding the adoption rate of new forms.

Based on the assumption that language change becomes apparent not only between different points in time but is -- on a smaller scale -- also observable between speakers of different generations, different social backgrounds etc. (see Section \ref{section:Factors} for a more detailed overview of the different factors), the \textit{apparent-time model} has gained popularity as a surrogate for real-time evidence for capturing change in language \parencite[cf.][]{Labov1963,Bailey1991,Cukor-Avila2013}. 
The concept behind the apparent-time model is the assumption that the variety used by a certain individual would signal the state of the language that they have acquired at a young age \parencite{Labov1999,Schilling-Estes2005}, as individual vernaculars are supposed to be less liable to change after adolescence \parencite[cf.][320]{Bailey2002}. Recordings of people born in different times and surveyed in the same study could feed apparent-time analyses, where the temporal depth is projected from the contemporary recorded state. 
However, since most studies in the research paradigm of sociolinguistics put their focus on the correlation between linguistic variants and social rather than geographic factors, studies typically took place at one location or city. The amalgamation of the two approaches started only in the 1980s, resulting in a field of research named \textit{socio-dialectology}. In German dialectology, \textcite{Mattheier1980} introduced pragmatic and societal aspects as integral parts of dialectology.
Apparent time evidence minimizes the variation that might arise from differences in sample populations, in elicitation strategies, and in the recording and presentation of data when evidence from an existing data source is compared to data from a new study \parencite{Bailey2002}. These properties brought popularity to the model in dialectology \parencite[for a review, see][64--65]{BeamanThesis2020}.

Unlike the approaches discussed in Section \ref{Approaches: geography in focus} that often treat geography as a constant, more recent research seeks to “un-trivialize […] the connection between language and space” and to take speakers’ perceptions and their “construction of language spaces through linguistic place-making activities” into account \parencite[][1]{Auer2013}. The traditional geolinguistic paradigm is therefore criticized for being too deterministic, neglecting important aspects of linguistic reality such as the  mobility of speakers, multi-varietal competences of individuals or the choice and systematic use of certain features to create regionalism. Cognitive aspects, such as salience, are not often considered in such studies. 

The idea that speakers’ awareness plays a central role with respect to the perpetuation or abandonment of linguistic features in contact situations was already put forward in the early 20th century by \citet{Shirmunski1928}.

To date, the concept of \textit{salience} is still controversial. One main question is whether salience can be determined objectively or whether it has to be considered a purely subjective category.\footnote{For a detailed discussion of the subject, see \textcite{Christen2014} and the other contributions in the same journal issue.} \textcite{Labov1972a} distinguishes between different degrees of awareness and their role in the process of language change, called indicators, markers, and stereotypes.
Building on \citegen{Silverstein2003} concept of “indexical order" and combining it with Labov's approach, \citet{Johnstone2006} assume three orders of indexicality (first-, second- and third-order indexicality; \citealt[cf.][82]{Johnstone2006}). 
In their research in Pittsburgh/USA, they examine how a set of linguistic features which speakers originally were not aware of (first order indexicality) undergoes a transformation, becomes available to speakers' attention and can be linked with locality (second order indexicality). In the last step, these features function as markers of social affiliation, finally becoming constitutive characteristics of the concept of “Pittsburghese” (third order indexicality). This process is called “enregisterment” by the authors \parencite[77]{Johnstone2006}.\footnote{Although these steps describe a diachronic development, it has to be noted, however, that not all features progress from first order indexicals to second or third order indexicals \citep[14]{Auer2013}.}

A closely related field of research that gained popularity through Dennis Preston’s work (see \cite{Preston1989}, \citeyear{Preston2005}), known under the umbrella term \textit{perceptual dialectology} or \textit{folk dialectology}, focuses on speaker-related factors and the perception of dialects/languages and their variation. Here, to gain a more thorough understanding of the dialectal variation and change, it is necessary to include ideas and beliefs of the speakers -- not only with respect to particular features but also regarding the geospatial structuring of the dialects as well as their evaluation. Contributions within the field of perceptual dialectology acknowledge that functioning communication channels may not be the only driving force in dialect change, but that speakers' perceptions and ideas about dialects and dialect boundaries as well as their evaluations play a crucial role (cf. \cite[][160--161]{Auer2004}, \cite[172]{Kristiansen2009}).
A commonly used research method is mental mapping, including dialect maps drawn by informants \parencite[for an account of the methodology and its application, see][]{Montgomery2013}. These hand-drawn maps can deliver insights into categorization principles used by non-experts, and they can provide useful background information for understanding dialect change. 

\section{Processes and Patterns}
\label{Section:P&P}

In this section, we first focus on contact processes of various intensities (Section \ref{Subsection:Processes}), moving from examples of contact between single speakers (idiolects) to contact between groups of speakers (dialects).
Then, we describe spatial patterns that result from these dialect contact processes and the linguistic changes they provoke (Section \ref{Subsection:Patterns}).


\subsection{Processes}
\label{Subsection:Processes}
What is considered dialect contact may range from a shop assistant talking to a customer at the counter to different groups of speakers settling in a newly built town. The various dialect contact scenarios can be classified by a number of factors, such as the duration of contact, its intensity, and the number of speakers involved. These three factors (duration, intensity, and number of speakers) can occur in different combinations, resulting in many different contact situations.
When dealing with contact-induced change, research focus rather lies on long-term accommodation, which can be defined as “the adjustments speakers make to become linguistically more (convergence) or less (divergence) similar to an interlocutor or to a social environment" (Chapter \ref{chap_accommodation}).
In this section, therefore, we focus on long-term contact, of various intensities. 
The number of speakers considered in dialect contact studies varies from single speakers or families \parencite[e.g.,][]{Ghimenton2013} to bigger samples, such as Britain and Trudgill's \parencite*{BritainTrudgill.2005} study on 81 speakers. In addition, we can distinguish between constant and temporary contact situations. 

Research on second dialect acquisition is important for understanding the mechanisms of single speakers' long-term accommodation to a majority dialect. Research usually involves individual speakers or small groups of speakers, such as families or speakers matching a set of sociolinguistic features, in constant long-term contact situations \parencite[22--51]{Siegel.2010}. Speakers who, for instance, move to a place with a differing, yet mutually intelligible dialect, acquire the recipient variety usually incompletely in the first generation. Their linguistic system is shaped by mixing the varieties and by making use of interdialect or compromise forms, forms which are neither found in the recipient nor in the source variety \parencite[117]{Trudgill1986,Wilson2019}.


Single speakers who are \textit{sedentary} \parencite[cf.][]{Britain2016Sedentarism}, but commute to another place, find themselves in a somewhat different contact scenario. They are in contact with both speakers of their own dialect and, temporarily, speakers of other dialects. 
A common consequence of such a contact situation is \textit{dialect leveling} during which highly local forms erode in favor of forms with a wider distribution. In other words: “linguistic variants with a wider socio-spatial currency become more widely adopted at the expense of more locally specific forms” \parencite[][193]{Britain.2010}. This process is also known as \textit{supralocalization} \parencite[][8-10]{Milroy.2002}, \textit{supraregionalization} \parencite{Hickey.2003}, or \textit{regional dialect leveling} \parencite[]{Kerswill2003} and it has been described as the spread of urban dialect features to rural dialects \parencite[]{BritainTrudgill.2005} and vice versa \parencite[]{Britain2010LangAndSpace}. The tendency of urban forms spreading, often associated with commuting and the patterns observed in the hierarchy of  settlements, is also addressed by \textcite{Trudgill1974} in the model of \textit{linguistic gravity}. Often the majority variant supplants other variants. However, other factors such as markedness, social or regional stereotyping, and salience seem to also have an influence on the choice of the variant present in a variety that has undergone leveling \parencite[195]{Britain.2010}. It has been suggested that, apart from leading to simplification of the emerging variety, for instance by reducing paradigmatic redundancy, it may also lead to the emergence of new features \parencite[13]{Williams.1999}.

The so-called \textit{tabula-rasa}-situation is the most extreme form of dialect contact, such as in the case of the colonization of New Zealand \parencite[]{Trudgill.2004}, the formation of new towns, for example Milton Keynes \parencite[]{Williams.1999}, or the reshaping of previously uninhabitable areas such as the Fens \parencite[]{Britain.1997}. Dialects engaged in such contact are often classified as high-contact varieties. \textcite[66-67]{Trudgill.2004} lists as high-contact varieties of English “colonial and/or urban and/or shift and/or standardized varieties which have a considerable history of dialect and/or language contact, and therefore show very many signs of simplification." In each scenario involving high-contact varieties, the contact situation has changed drastically due to colonization, urbanization, changes of migration and mobility patterns of the speakers or also due to “greater socioeconomic interaction" \parencite[][192]{Croft.2000}. Hence, speakers of various dialects who have previously not or hardly been in contact, suddenly form a speech community and new dialects arise through the process of \textit{new dialect formation} \parencite[]{Britain.2009,Kerswill.2010,Schreier.2017}. New dialect formation passes four stages \parencite[]{Trudgill1986}: 

\begin{enumerate}
    \item \textit{leveling} of marked features --- markedness can be qualitative, for instance if a feature is stereotyped, or quantitative, for instance if a feature occurs less frequently than the one in whose favor it is leveled
    \item \textit{simplification} of the morphological system and of the constraints on variation on all linguistic levels
    \item development of \textit{interdialect formation} due to contact between adult speakers of various varieties
    \item \textit{reallocation} in which varying forms of one variable are refunctionalized
\end{enumerate}
This process results ultimately in the emergence of a \textit{koine}.\footnote{For the differentiation between immigrant and regional koine, see \textcite[175f.]{Siegel.2001}. For more details about koineization, see Chapter \ref{chap_contactlanguages}} 

Two further contact-scenarios to be discussed are the “casual" contact between neighboring dialects or simply neighboring speakers, and varieties lacking contact, so-called isolated varieties.

Casual contact is not due to any recent changes in the speakers' behaviour, but has been on-going and constant. It is usually not researched by contact dialectology, but by variationist approaches, because it concerns contact between idiolects and not an idealized form of a homogeneous dialect. There are no two identical linguistic systems which are in contact with each other, even if the communication involves two speakers from the same village. Variationists investigate how this contact between individual speakers affects the linguistic systems engaged and its effect on the spread of variants in one idiolect within a dialect or a dialect area. These questions are discussed in modern sociolinguistics under the notion of \textit{transmission}, an “unbroken sequence of native-language acquisition by children" \parencite[346]{Labov.2007}, in which the differences and similarities of phylogenetically related languages (or dialects) stay stable, and \textit{diffusion}, contact-induced “transfer across branches of the family tree" \parencite[347]{Labov.2007}.\footnote{\textcite{Labov.2007} elaborates on the linking between transmission and diffusion and the two models of language change in historical linguistics, the family tree and the wave model.} This is strongly related to general theories of language change, such as the two traditional models in historical linguistics: the family tree model, in which linguistic features are transferred down the generations, and the wave model, in which features spread (in the form of innovations) from one variety to the next in space with decreasing intensity. In contact dialectology, this is embedded in the research into accommodation. 

An isolated variety is by definition in low or no contact with other varieties.\footnote{Isolation can be defined spatially, socially, and individually  \parencite{Schreier2009, Schreier.2014} The overall concept of isolation still lacks an operationalization \parencite[353--355]{Schreier.2017} and it needs to take perceived isolation into account. Speakers might perceive their variety as more or less isolated than estimated by a linguist \parencite[]{Montgomery.2000}.
For more details on isolation, see Sections \ref{subsubsection:spatial patterns} and \ref{subsection:geogfactors}.}. \textcite[][]{Trudgill.1992} defines the prototype of an isolated language as a small speech community located in a geographically isolated area, with few L2 speakers and little contact with speakers of surrounding varieties. These communities are characterized by dense social networks, a high social stability and large amounts of communally-shared information \parencite[146]{Trudgill.2011}. In such communities, \textcite[6]{Trudgill.1996} expects language change to be slower and due to their lack of contact, less subject to “language change leading to simplification."\footnote{\textcite[349]{Manczak.1988} notes that “it always was evident to linguists that dialects spoken in isolated areas like islands, mountains, etc., show an archaic character."} However, not only is complexity preserved in isolated varieties, but the linguistic system can even be complexified due to the isolation of a variety linked with certain sociolinguistic features like a small community with a dense social network \parencite[]{Trudgill.2009}.

\subsection{Patterns} 
\label{Subsection:Patterns}

Spatial patterns in the dialectal landscape, their presence and changes are indicative of linguistic processes, such as new dialect formation or dialect leveling in progress.
As  spatial counterparts of the aforementioned processes, areal patterns of dialects can be investigated through static representation of the areas themselves and, conversely, through the characterization of the interfaces between areas, their boundaries.

\subsubsection{Spatial patterns of dialect change}
\label{subsubsection:spatial patterns}
\begin{sloppypar}
Speakers experience variation in language most prominently through differences present in geographic space. This variation emerges as a result of language change processes, and is ubiquitous: language does not converge towards stability or a goal. Thus, the spatial patterns and the perceived state of language are a mere snapshot of a changing linguistic landscape. 
\end{sloppypar}

More precisely, the perceived spatial patterns of the variation are the distribution of different existing variants, and the dynamic patterns of innovation, which hint at patterns of language variation being strongly related to patterns of contact \parencite[cf.][]{Nerbonne2007,Lee2014}. Some of the observed spatial patterns and areal constructs in language are more stable and stay around longer while others dissolve more easily in the process of language change. Over time, local dialectal varieties, present in a relatively small area such as a village or a valley, may diverge from surrounding varieties, or they may converge to nearby varieties, a regiolect or the local standard language \parencite[cf.][]{Auer1996}.

The \textit{density} of contact is often identified to be the driver of contact-induced language change \parencite[cf.][414]{Bowern2013a}. Thus, at a micro scale, well-connected, central people (with higher prestige) are assumed to drive language and dialect change \parencite[e.g.,][]{Fagyal2010,Trudgill2014,Burridge2018} and, similarly at a macro scale, central communities, cities with a high contact density across different idiolects drive linguistic change. Linguistic innovations catching on later with individuals who are less central and more isolated are also reflected in the city – hinterland networks with less well-connected, more isolated areas maintaining original variants and adopting innovations later \parencite[e.g., Elfdalian in Sweden, cf.][]{Sapir.2005}.

The potential for dialect change is often investigated based on the spatial patterns of dialect variation. The presence of a larger amount of variation (heterogeneity) within an area or a speaker's language usage is often regarded as an indicator for dynamism and linguistic change, although longitudinal studies are needed to confirm this connection \parencite[cf.][]{Stoeckle2016a}. 
Notably, more dynamism, with certain variables changing faster and the adoption of innovations being quicker, while isolated, mostly rural areas seem conservative.
Besides, spatial patterns of the dialect change potential may also depend on the \textit{homogeneity} of the dialects used. Sprawling urban areas with masses moving in from different dialectal areas (including different sociolinguistic varieties) might develop a more heterogeneous local dialect landscape \parencite[such as \textit{urban dialects} ---][]{Labov1966, Britain2012, Proll2019urbanareal}, which often acts as the cradle of dialect change and innovations. As a consequence of not being exposed to a colorful variation, rural areas that lack a significant influx (migration or mundane contact through commute) from other areas may stay or become more homogeneous and resistant to change.
The presence of more connected or linguistically more isolated regions is also highly dependent on the mobility of the population \parencite{Britain2013b}. In turn, mobility itself often appears to be a self-reinforcing process based on the spatial patterns of wealth and economic prosperity.
The spatial manifestation of these processes leads to the formation and presence of areas more susceptible to language change, and of linguistically conservative dialect “strongholds", sustaining older forms that might be considered archaic elsewhere.  
Furthermore, state-level language policies \parencite[e.g.,][]{Valls2013} and the presence or absence of strong dialect conservation trends might spur areal trends in the general prestige of dialect usage, such as the case of Elfdalian \parencite[cf.][]{Sapir.2005} which boasted with a lot of archaisms that Swedish lost a longer time ago.

\textit{Relic areas} can form at different levels of variation, be it a language, larger scale dialect area or an individual variable. A linguistic variety (e.g. language, dialect) can be present in two or more regions, separated by an area in which a different, or opposing, form  dominates. Such a pattern might indicate a late stage in the displacement of a formerly widespread variety following a migration of speakers (e.g., Walser dialects within Alemannic German) or an innovation within a variable \parencite[e.g.,][]{Lizana2011}.

\subsubsection{Areal and linear constructs}
\begin{sloppypar}
Dialect variation displays striking spatial patterns. However, defining areas within the dialectal landscape and drawing the boundaries between areas of different constructs is almost always difficult. 
The generally transitional nature of language change contrasts with the picture painted by sharp linear boundaries, often used in dialectology.
The need for such classification is fueled by the importance of dialects in group identity formation.\footnote{The fact that they are often named after certain areal features (e.g., Wallis German, Gail Valley Slovene or Bergamasque Italian) also shows the spatial nature of dialects and their importance in identity.}
One driving force behind linguistic classification, group formation and area definitions appears to be the inherent human need for categorization, which is broadly discussed in cognitive psychology \parencite[for a cognitive linguist's overview, see][]{Lakoff1987}.
\end{sloppypar}


Areal constructs in dialects are often described (qualitatively and quantitatively) at different levels of granularity, both spatially and in terms of attributes. 
Spatial patterns are present at attribute levels ranging from individual linguistic characteristics to entire grammatical systems.
The number of variants expressing a particular concept may vary depending on the linguistic level, from a few (such as in the case of syntax, e.g., the expression for ‘the ice begins to melt' in  the Syntactic Atlas of Swiss German Dialects, \citet{Bucheli2002}) up to hundreds (such as lexicon, e.g., ‘snail' in the Linguistic Atlas of Japan, NLRI,  \citeyear{NLRI.}) of dialectal variants. Often, few dominant variants are present in larger areas, while less frequent variants are confined to smaller areas. Areas of main variants often comprise regions where the more frequent variant is used interchangeably with less frequent regional variants that are locally more characteristic.

Spatial boundaries are very often perceived in dialectal variation. 
As language changes gradually in the temporal dimension, a logical assumption is that it is possible to capture the gradual nature of this change in the spatial dimension as well, within the spatial patterns of the diffusion of innovations. One person might switch to the new form immediately, one might use both variants, and one might not change at all (depending on features of the speaker that are broadly discussed by, e.g., sociolinguistics).
Dialect atlases and large scale surveys have attempted to unravel the granularity of spatial variation. Notably, dense networks of survey sites have led to latent fuzzy looking boundaries.

The uncertainty and fuzziness present in the spatial variation within a language spawned a need for defining boundaries quantitatively. Dialectometry often investigates linguistic variables in an aggregated manner, to characterize the multidimensional nature of dialects and describe overall spatial patterns \parencite[e.g., ][]{Seguy1971, Goebl1982, Nerbonne1999,   Szmrecsanyi2012}.
Based on the aggregation of dialectal differences across survey sites, researchers established quantitative methods to reveal distinct areas and, conversely, to show their interfaces. Most research, however, has focused on the homogeneity of areas and class affiliations, with boundaries viewed rather as implicit by-products  \parencite[e.g.,][]{Daan1969, Heeringa2001, Heeringa2004, Rumpf2009}. At the same time, quantitative characterizations of the strength of such linear constructs are scarce \parencite[e.g.][]{JeszenszkyPaper2}, mostly due to the coarse spatial granularity  of the data available.

It is possible to position characterizations of spatial dialectal boundaries between two extremes: watertight, strict, linear boundaries on one end of the scale and completely fluid, fuzzy boundaries of gradual nature on the other, which are more properly regarded as transition zones. Researchers often characterize areas and boundaries in relation to the transitional nature of change. Boundaries at the level of an individual variable are often represented by clear-cut isoglosses, which imply an assumption of homogeneous variant usage at each survey site, thus on the two sides of this isogloss (see Section \ref{Approaches: geography in focus}). If we consider isoglosses analogously to boundaries in a dialect continuum as \textit{gradual} transitions between two core dialect areas, isoglosses can be viewed as ``sharp transitions" between the dominance zones of variants. Often, however, several related variables and their isoglosses \parencite[][]{Seiler2005, Glaser2006b,Stoeckle2018, Willis2019} are aggregated in order to investigate the distribution, transition and different levels of grammaticalization of certain phenomena.

Patterns aggregated from several, coinciding and nearby isoglosses, so-called \textit{isogloss bundles}, were traditionally often used to quantitatively account for dialect areas at different spatial and attribute granularities,
by highlighting boundaries between (mostly) homogeneous dialect areas  \parencite{Handler1982}. It has often been noted, however, that isogloss bundles do not fulfil all expectations as a means of delimiting dialect areas, as individual variables tend to show different patterns of regional variation \parencite[94--103]{Chambers2004}.

Any spatial classification and grouping within language by sharp boundaries may be, however, inherently flawed, as we determine boundaries within the construct of a language, which essentially varies on a continuous basis across people and in time. Clear, linear boundaries are useful, however, for the overall visual interpretation, especially at smaller geographic scales. Data in collections is potentially fuzzy due to its sparse and often biased nature, which is partly due to the assumption of local internal homogeneity customary in traditional dialect atlases. Because of this, simplifications like identifying boundaries are often needed to make overall interpretations. Modern dialectometry attempts to resolve the boundary issue by considering as many relevant variables as possible to characterize the spatial patterns within a language area, making it possible to quantitatively warrant clear-cut boundaries or gradual transition \parencite[e.g., ][]{Seguy1971, Goebl1982, Nerbonne1999, Nerbonne2009DataDriven, Burridge2018}. Corpus-based dialectometry usually weighs findings about different phenomena with their relative frequency (relating it to salience in real-life usage) \parencite[e.g., ][]{Szmrecsanyi2011, Wolk2018}.

There is agreement in dialectology, however, that linguistic variation is gradual, not abrupt, despite most discussions of dialectology in textbooks dealing with isoglosses and dialect continua side by side, without addressing their incompatibility \parencite[][105]{Chambers2004}.
Experimental research attests that dialect areas, be it larger-scale areas or areas of abutting variants of a single variable, rarely have clear-cut boundaries and are mostly characterized by a transition towards the dominance zone of neighboring varieties that is gradual to a certain degree \parencite{Kessler1995,Heeringa2001,Chambers2004, Pickl2012}. Where such transitions take place, wider or narrower \textit{transition zones} are found, marked by the mixing of co-occurring variants (see also Section \ref{Approaches: geography in focus}). Conversely, transition zones are often regarded as zones of ongoing change and indicative of the patterns of contact, also in relation to linguistic features beyond the one that is mapped. Moreover, transition zones can appear as autonomous areas on their own, with the grammaticalization of both variants present, possibly at different stages regarding different contexts \parencite[e.g.,][]{Seiler2004, Willis2017}. 

Transitional patterns often seem to correspond to further underlying factors hindering or promoting dialect contact across the areas (e.g., geographic factors) and the sociodemographic groups using them. Notably, the simultaneous presence of multiple variants often appears to be associated with groups with different characteristics (age, class, educational background, or dialectal attitude) \parencite{Willis2017}.

\section{Factors} 
\label{section:Factors}

In this section we discuss the main factors that affect the realization of dialect contact by facilitating or hindering it, thus contributing to  dialect change, which in turn may contribute to the modification and formation of  dialect areas. We describe factors related to language and the speakers themselves, after which we detail interactional and geographic factors.

\subsection{Linguistic factors}
Although there is no consensus about this topic, historical linguists have speculated that rates of change are different at various linguistic levels \parencite[e.g.,][1694--1695]{Longobardi2009} and at finer attribute granularity of structural linguistic features  \parencite[][]{Dediu2013}, with syntactic variables often changing more slowly as opposed to lexical or phonological ones.

Since dialect contact is ultimately based on face-to-face interactions between speakers of mutually intelligible variants, its fundamental mechanism  is accommodation \parencite[cf.][]{Trudgill1986}. Ruch \& de Benito Moreno (Chapter \ref{chap_accommodation}) discuss a number of linguistic factors regarding accommodation. An additional factor relevant to the formation and structure of dialect areas is the varying \textit{degree of relatedness} between different dialects of one language.

Dialects are \textit{per definitionem} related and often grouped based on shared features. Higher structural similarity makes dialects more prone to contact-induced change within their grouping \parencite[][74--75]{Trudgill1983}. Hence, the observation by \citet[][413]{Bowern2013a} should be extended to include dialect group boundaries: “We can observe that most linguistic changes spread easily through speech communities, less easily (but still fairly easily) across dialect boundaries where speakers are in contact with one another, and less easily still across language boundaries".




\subsection{Speaker-related factors}

A central concept for the notion of dialect is geographic space (cf. the definition of “dialect" in Section \ref{Intro:Dialects_and_dialect_areas}). However, the relationship between geographic and social factors regarding the concept of dialect varies according to research tradition. While in the German, Italian or French research context most consideration is given to space, and dialects are defined based on their spatial distribution, in the Anglo-American tradition the social position of the speaker also plays an important role \parencite[cf.][1436f.]{Mattheier2005}. Besides, in most modern societies a connection is assumed between dialect usage and certain social parameters, and due to their interaction, the two cannot always be separated. For instance, in many languages, speaking a non-standard variety is associated with low prestige and lower social classes.
With the increasing number of studies analyzing social influences on dialect use and the necessity of implementation in empirical research, some parameters have emerged which will be discussed below. 

\subsubsection{Age}

Age is commonly regarded as one of the most important factors with respect to language use and change. It is assumed in the apparent time paradigm \parencite[cf.][]{Bailey2002} that language norms and forms are adopted at a relatively young age and, depending on the linguistic level, the rate of change will decline by age.

However, it is assumed that it is not biological age \textit{per se} which is decisive. \citet{Mattheier1994} emphasizes an age-related generation model that is sociologically sensitive to the use of varieties, whereby the age-related change in everyday socio-communicative relationships seems to be responsible for changes in the spectrum of varieties. Important phases are the pre-family phase, which is often characterized by training processes, the (the phase of) entry into professional life. This often coincides with the time when children are raised and is characterized by a tendency towards supra-regional language varieties, and the retirement phase, in which official and formal language contacts decrease.

Age is used as a predictor and indicator against the rates of adoption in studies of ongoing dialect change \parencite[e.g.,][]{Willis2017}. As younger people are assumed to have more connection with other young people outside their home community, age also tends to indicate spatial patterns of contact and isolation.

\subsubsection{Social position}

In relation to Labov’s \parencite*{Labov2001} question about the leaders of language change , the \textit{social position} of the individual plays a central role. This can be measured using various parameters that are of different relevance depending on the social profile, often interacting with one another. In the Anglo-American tradition, the concept of \textit{social class} was used, and corresponding varieties were defined as \textit{sociolects} \parencite[190]{Dittmar1997}. One problem, however, is that this term has a strongly evaluative character and, moreover, the definition of social classes does not always turn out to be unambiguous, making it difficult to assign individuals to classes. The level of \textit{education} and the type of \textit{profession} have become proven indicators that are more easily implemented in empirical studies. 

With regard to profession, \citet{Mattheier1994} differentiates between \textit{script-ori\-en\-ted} and \textit{craft-oriented} professions as well as between those with and without \textit{authority to issue instructions} (Germ.: \textit{Weisungsbefugnis}). Hierarchies can be derived from these parameters that are closely related to the social position of the speaker and thus also to the corresponding social prestige, which can be decisive for accommodation and change in the event of contact.

\subsubsection{Religion}

While the factor of \textit{religion} or denomination certainly played a more important role in relation to dialect contact in earlier times, in most modern societies it is of secondary importance and is therefore not taken into account in most studies. In principle, religion can play both an isolating and a unifying role. On the one hand, distinct religious groups may be isolated from one another, based on areas with different predominant religions within a language area, or through the prohibition of intermarriage \parencite[e.g.,][]{BucheliBerger2014}. On the other hand, religious communities can integrate people using different dialects or languages, catalyzing change. An example of this is Amish Shwitzer, a mixed language spoken by a group of very conservative Amish in Indiana. It has features from both Bernese Swiss German and Pennsylvania Dutch, evolving through intense contact between two groups of the Amish \parencite{hasse_in_press}.

 The study of \citet[][16]{manni2006extent} finds that, measured at the level of municipalities in the religiously segregated Netherlands, religion does not correlate with dialectal distribution. Yet indirectly, the influence of this factor is visible today in several contexts, since religious affiliation often corresponded to the extent of rulers' territories or other administrative areas, which are still reflected in political units today and, thus, may influence dialect contact.\footnote{In the context of complex modern societies, ethnicity and migration, religion certainly plays an important role today as well. These are phenomena of multicultural contact that are primarily important in cities, but in relation to dialect areas they seem to play a subordinate role and are therefore not discussed further here.}\\

\subsubsection{Gender}

The \textit{gender} factor has often been considered in studies as a factor to explain language variation and change. However, it has been assigned different meanings, which has led to controversial representations of gender-specific language use \parencite[228]{Diercks1986}. \textcite[367]{Labov2001} states that women orientate themselves more strongly to linguistic norms if these are overtly prescribed than men. Therefore, when new prestige variants spread, these are more likely to be used by women. In other studies, however, it is assumed that women are preservers of the dialect \parencite[see, e.g.,][299]{Sieburg1991}. Overall, there are many indications that gender differences are primarily related to social position and aspirations for social advancement (which vary around the world). Apart from its relevance as a determinant of linguistic variation, gender has been researched in the framework of social constructivism \parencite[368]{Queen2013}, and it has been argued that it is not to be seen as a “static social category" \parencite[383]{Queen2013}.
Because of its controversial status, \citet{Mattheier1994} suggests to leave the gender factor in representative surveys to random distribution.

\subsection{Interactional factors}

Interactional factors between individuals have lately attracted more attention in the study of language/dialect contact. One focal approach  has been \textit{linguistic accommodation}. 
The concept was introduced by the work of \textcite[e.g.,][]{Giles1973b, Gilesetal1991, Giles2008} in order to account for the fact that speakers sometimes adapt their way of speaking to their interlocutor in order to gain approval, which is closely connected to the social status of the speakers \parencite[335]{niedzielskietal1996linguistic}.
\citet{Trudgill1986} distinguishes between short-term and long-term accommodation, with the latter potentially leading to modifications or alterations in the speech of a group of speakers.  
For more details on the effects of accommodation, see Ruch \& de Benito Moreno (Chapter \ref{chap_accommodation}). 

Another, related approach is suggested by \citet{SchmidtHerrgen2011} in their theory of linguistic dynamics (\textit{Sprachdynamik}). The central idea within this theory is the concept of synchronization, which is defined as “the calibration of competence differences in the performance act, [...] [resulting in a] stabilization and/or modification of the active and passive competencies involved" \parencite[212]{Schmidt2010}. They distinguish between three levels of synchronization, with “microsynchronization" referring to the processes taking place in single interactions.\footnote{The other types which go beyond single communicative events and may potentially result in dialect change are called \textit{meso-} and \textit{macrosynchronization}.} In contrast to linguistic accommodation, they assume prestige of a variety or the social status to play only a minor role; it is rather “the desire to be understood, or at least not misunderstood" \parencite[212]{Schmidt2010} which is considered the driving force behind linguistic convergence. 

\subsection{Geographic factors}
\label{subsection:geogfactors}
Whether considering geography as “geographic distance or as the basis of an areal division among varieties, it certainly should not be understood as a physical influence on language variation, but rather as a useful reification of the chance of social contact” \parencite[][14]{Nerbonne2013a}.

In this subsection we present factors most often investigated in dialectology and dialectometry, along with quantitative methods to measure these factors, where relevant.

\subsubsection{Physical proximity}

Physical proximity -- at least until the middle of the 20th century -- has been the only way to maintain close contact between people.\footnote{The effect of media on everyday language is, in fact, contested \parencite[cf., e.g.,][]{Trudgill2014}.}   Therefore, geographic factors seem to determine \textit{contact potential}, thus playing a crucial role in the areal structure of dialectal variation and change, by constraining -- or conversely, facilitating -- people's movement.
The connection between the ease or difficulty of transportation and the spread of cultural artifacts  \parencite[cf.][]{Hagerstrand1952} has become one of the most important explanations of dialect diversity as well. 
Analogously, the connection between topography and dialect diversity is often intuitively assumed due to the cultural difference observed in the presence of such topographic features that hinder \textit{transportation}, in the broadest sense, and thereby contact \parencite[][5]{Nichols2013}.
The effect of geographic factors has been quantitatively tested in numerous studies, and has shown a predominant influence on dialect variation in large-scale, quantitative studies (for an overview, see \cite[][253--255]{Wieling2015} and \cite[][24--26]{Jeszenszkythesis}).

\subsubsection{Isolating features}

\textit{Natural isolating features}, such as those of a topographic nature, are often thought to pose obstacles for language contact. “Sharp" obstacles, such as mountain ridges and rivers, can be modeled as lines separating groups within a language area, such as the river Lech \parencite[][29-33]{Pickl2014} or a range in the Swiss Alps \parencite[][]{Jeszenszky2017}. Rugged terrain, dense vegetation and harsh climate conditions can influence contact to a remarkable degree, isolating people living within or separated by such areas. For example, Dogon dialects in Mali are separated by rocky escarpments \parencite[][]{Moran2013}, the effect of former marshlands in the Fens of Eastern England on dialects can be traced today \parencite[][218]{Britain2010a} and impenetrable forests in Amazonia make rivers the main media of communication \parencite[][]{Ranacheretal2017identifying}. 
Besides, territorial disputes or hostile inhabitants (guerrillas, drug lords) may also render areas difficult to traverse. Anthropogenic modifications of natural pathways, that is, improving transportation infrastructure or, conversely, creating obstacles to the free movement of people (such as national borders) influence contact potential crucially. 

\subsubsection{Realized contact and its quantification by surrogates} 
It is not possible to quantify all contact occurring between members of communities. Besides, all factors exert their influence in an overlapping fashion, strengthening and weakening each other. To address this difficulty, dialectology attempts to account for the effect of contact by using surrogates for calculating contact potential.
It is not straightforward to model spatial artifacts (e.g., areas and linear features that are difficult to traverse) as factors of contact or isolation, because there might be various potential reasons for actually not \textit{wanting} to traverse them. Thus, for the characterization of dialectal variation, it is fruitless to consider \textit{contact potential} without the \textit{realization of contact}. 

The motivation for non-sporadic travel over relatively long distances (the concept of ``long" distance is, of course, related to infrastructure and has therefore changed throughout history) is, most importantly, the economic or social interest of the traveler. Individual migration and commute is typically driven towards market and school towns, trading hubs and places with abundant working opportunities. Vice versa, the lack of such interests keeps outsiders away from certain places, contributing to potential linguistic isolation. Motivation for contact has also been shown to correlate with actual dialect similarity, through implicitly measuring the motivation for contact by the intensity of trade \parencite[][]{Falck2012, Falck2016, Lameli2015}.

The most intuitive predictor of potential contact between dialect data points is
\textit{Euclidean distance}: the shortest distance between two points, also referred to as geographic distance or linear distance. As it is easily calculated, Euclidean distance has been the predictor variable most often used in dialectometric studies for explaining linguistic distances \parencite[e.g.,][]{Seguy1971, Heeringa2001, Nerbonne2010a, Hadj2017}. 

\textit{Travel times and travel distance} are assumed to better correspond to the potential of people to meet, as they incorporate the isolating factors and obstacles present in the area of interest. Some studies \parencite[][]{Gooskens2004, Jeszenszky2017} show a strong correlation between travel times and the spatial distribution of linguistic distances, while others \parencite[][]{VanGemert2002, Stanford2012, Szmrecsanyi2012} do not confirm this hypothesis. Calculations with modern travel times might also be biased, not representing the historical routes of contact that have influenced dialect change for centuries and led to the contemporary state of dialectal variation. Although  travel times can be obtained today from open source online routing systems \parencite[e.g., \textit{osrm} --][]{Giraud2019}, creating a travel time database, especially a historical one, is a tedious process for most places.

The (historical) potential of contact can be further calculated in \textit{cost models} where different isolating factors (such as topography or boundaries) are represented as weights. Using cost distance models, \textcite[][]{Haynie2012} compared California Miwok languages, taking elevation, vegetation, surface water, and watershed boundaries into account. \textit{Hiking distance} approximates the most natural routes of communication, which therefore usually corresponds to historical routes. It is calculated along least cost paths that are based on a digital elevation model and a travel speed model \parencite[e.g.,][]{Tobler1993}. Hiking distance is a better predictor only in situations where topography significantly impacts the distributional patterns in dialects and it deserves to be noted that different types of land cover or natural barrier, such as dense vegetation, water bodies, and glaciers have to be taken into account beside an elevation model \parencite[cf.][]{Derungs2019}.
Especially in larger distances, the role of topography and infrastructure seems to fade away, and in aggregate area studies the explanatory values of the aforementioned distance measures converge towards each other  \parencite[][]{Jeszenszky2019}, due to the overwhelming proportion of indirect contact.

\subsubsection{Commute, mobility, migration}

Displacement of people in different contexts impacts dialect contact substantially.
Non-sporadic short-term contact, such as commute, depends on the infrastructure, the availability (including physical connections, such as bridges, ferries, or roads), safety and cost of transportation, and the motivation to travel; in short, the mobility of people. Thus, political, economic, and topographic factors play a role in the network formation of potential contact, with self-reinforcing processes often driving infrastructural changes. With the emergence of globalization, people's increasing mobility can be observed throughout the last few generations. With “the normalization of long-distance commuting, labor market flexibility and the consequent geographical elasticity of family ties and other social network links, supralocal functional zones are probably larger than ever before” \parencite[][20]{Britain2010LangAndSpace}, causing dialectal changes more than ever before \parencite[cf.][]{Sayers2009thesis}. 

Migration and commute among communities, and therefore, their effect on dialects, are often modelled analogously to gravity \parencite{Trudgill1974}, with the weight in the gravity model replaced by some surrogate, such as population, that measures the impact or the (economic) importance of communities. Statistical data on commuter balance, thus, presents itself as a potential metric to which language change can be compared.

Migration (see also Section \ref{Subsection:Processes}) can occur sporadically or \textit{en masse}, with different effects. For example, migration due to marriage is present all over the world, impacting language in a sporadic manner. Since women are more often displaced, and mothers usually have a greater influence on the children’s dialect, these displacements may cause  sporadic introduction of innovations from the mother’s former community into the new one  \parencite[e.g., ][]{Stanford2012}.

Speakers in minority groups and their offspring tend to assimilate to the majority variation in a short time.
In contrast, the presence of a critical number of speakers (e.g., through \textit{en masse} immigration) may retain original dialectal variety \parencite[22]{Fishman1966} and might exert a greater influence on the language usage of the majority. Masses migrating to a certain location from multiple different areas and displacement of a critical mass of dialect speakers from one area to another (e.g., due to conflicts) can result in new contact situations with different cross-pollinations taking place between local and incoming variations. When people from a mix of origins (but speaking the same language) suddenly settle at a location, leveling of marked features and ultimately koineization of dialects occurs.\footnote{The process of koineization is detailed in Section \ref{Subsection:Processes}. For pidgins, creoles and the emergence of new languages, see Chapter \ref{chap_contactlanguages}.} Examples include Fiji Hindi \parencite[][]{Kerswill2003}, which is formed through the influx of speakers of different Hindi dialects to Fiji, urbanization in the UK as a consequence of the industrial revolution \parencite[][]{Britain1999}, bringing rural population to cities, or on Japan's northernmost island, Hokkaido \parencite[][]{Kleander2018}, settled by speakers of  several different Honshu dialects.     

\subsubsection{Center vs. fringe situation} 
If we accept the assumption that linguistic innovations start to spread in high-contact, central population groups, fringe situations can play a role in dialect contact not only in the sociolinguistic but also in a  spatial context.

Even if an innovation emerges elsewhere first, it is often larger populations with a lot of contact potential (major cities) that drive large-scale diffusion \parencite[][622--625]{Britain2002}. At the same time, fringe situations have a significant role in retaining dialectal forms before innovations (\cite[][349]{Manczak.1988}, \cite{Schreier2009}). Examples include Wallis German in Switzerland \parencite[topographic fringe: isolated by mountains from other dialects, cf.][39]{Moulton.1941}, Amish varieties in the US \parencite[e.g.,][818--821]{Louden2020} and areas with a strong caste system in India (socioeconomic fringe), Elfdalian in Sweden \parencite[economic fringe, see][]{Sapir.2005} or the western and northern extremes of Japanese islands \parencite[geographic fringe, see][]{Abe2018}, similarly to individuals with less contact in the sociolinguistic model on preserving older forms by \citet{Fagyal2010}. It has to be noted that spatial fringe areas may have contact to other languages, introducing a different confounding factor \parencite{Steiner2021}.

\textcite{Fagyal2010} scaled the gravity effect introduced by \textcite{Trudgill1974} down to the personal level, and concluded from their agent-based models that people with many contacts are the drivers of language change. \textcite[][]{Burridge2018}, using an interaction model and based on the laws of surface tension, proved that both wave-like spread and hierarchical diffusion observed by linguists may be understood in a unified way. His  model showed that jumps of linguistic forms between cities (representing hierarchical diffusion)  are followed by a slower evolution, resembling movements driven by surface tension (representing the wave-like spread of innovations). He also showed how population mixing and long-range interactions can destroy local dialects either by overwhelming local linguistic variants when immigration is above a critical level, or by speakers that are weakly embedded in their social network.


\subsubsection{Spatial barriers}

Different spatial barriers contribute to several types of isolation between communities. Boundaries in space, similarly to the extents of geographic land coverage, can be investigated at different granularities, for example, a forest posing an obstacle between two villages, or a  country border between a majority language and its minority speakers in another country. \textcite[][63]{Gerritsen1999} concludes that “political factors can have a strong effect on dialect change.” 
Man-made boundaries, although they are often results of arbitrary decisions, often overlap with the natural isolating features mentioned above. 

For dialect contact, (historical) permeability of barriers has a high importance.
Permeability, that is, the contact potential across the barrier, however, does not always correspond to the realized cross-boundary contact, hampering the quantification of boundaries' role in contact. According to \textcite{Britain2010a}, the variation of language usage that has patterns in space is the outcome of routine dialect contact. The boundaries may physically be very permeable, but routinized paths might still tend not to lead people to cross them, due to the perceived separation effect or actual large differences regarding political, economic and other factors on the two sides. De Vriend et al. \parencite*{deVriend2008} show how the intelligibility within the dialect continuum of Kleverlandish along the Dutch-German border has decreased significantly despite the border becoming more permeable. These processes also seem self-reinforcing, similarly to the gravity-like effects mentioned in Section \ref{Subsection:Processes}.

Perceived boundaries within countries, such as cultural, religious, denominational and tribal ones, might also often mean a limit to the routinization of movement across them. The historical importance of such boundaries, especially if they become administrative boundaries, may also have a longer lasting perceptual effect on dialect areality, through generating identities, affection, refusal or prestige (e.g., refusal of a certain dialect, refusal to be associated with a certain area). \textcite{Derungs2019} tested the effects of several administrative and denominational boundaries on Swiss German syntax while \textcite{Valls2013} investigated border effects on Catalan.

Communities of speakers of the same language are often not only separated by national boundaries, but also speakers of different languages in between (thus forming a \textit{Sprachinsel}), posing a different level of hindrance in the motivational and potential components of dialect contact \parencite[e.g., diaspora communities of any language around the world, such as Arabic speakers in Central Asia, see][]{Fischer1961}.

\section{Conclusions} 
\label{Section:Conclusion}

The aim of this chapter was to give an overview of the notion of dialect contact within the larger context of language contact. To this end, we addressed the issue of how to define the object of our investigation, i.e. what is a \textit{dialect}, and what is a \textit{dialect area}? Within modern linguistics, the study of dialects is one of the longest-standing traditions, dating back to the second half of the 19th century. Since then, our societies have changed fundamentally, which in turn had a strong impact on the linguistic situation in most countries and speech communities. At the same time, the approaches and methods to study language variation have developed in different research traditions in various ways, referring to different concepts by the term \textit{dialect}. The level of granularity also emerges as a difficulty with respect to the definition of our research object, involving questions around the geographic scope of a dialect, for instance how to draw boundaries between geographically adjacent dialect areas. Considering these difficulties, our chapter followed the Continental European tradition and used a minimalistic definition of dialect as a variety spoken at a geographically defined place. 

Dialects are generally regarded as historically closely related varieties of a language, often (but not necessarily) united under a common standard variety, or, at least, a common norm. Although dialect contact can occur between dialects which are geographically distant from each other (e.g., in the case of migration), dialect contact is mostly regarded and treated as contact between geographically close or adjacent varieties. Therefore, contact generally takes place  between mutually intelligible varieties, which can lead to very fine-grained modifications in the language systems (also called micro-variation). Since dialects are often ideologically charged, associated with a certain prestige and a common identity, various social or demographic factors can also be included in the investigation of dialect contact and variation.

In order to provide a thorough overview of the field of dialect contact, we discussed different dialectological approaches, including more traditional ones focusing on base dialects, as well as studies following the sociolinguistic paradigm, taking different aspects of modern complex societies into account. We also discussed the processes that may take place in different scenarios of dialect contact, which can be classified by factors such as the duration of the contact, its intensity, and the number of speakers involved. Since dialects can be characterized primarily by their geographic extent, they display certain spatial patterns which may change through contact. A distinction is often made between urban, linguistically heterogeneous, more dynamic regions and rural, linguistically homogeneous, more conservative regions.

This has led to the assumption that cities are the drivers of dialect change and may be regarded as the centers of leveling and diffusion. However, in modern societies where large parts of the population have access to digital communication and media and therefore are able to at least virtually participate in linguistically complex speech communities, the distinction between urbanity and rurality may play a less important role than it used to. Besides, cities may still be cultural melting pots, but migration and mobility are certainly aspects of rural life as well. All in all, there are a lot of linguistic and extra-linguistic factors that may potentially influence dialect contact and its outcome, which we discussed in Section \ref{section:Factors}.  

Contact dialectology, through approximately 150 years of development within different fields of linguistics, has itself undergone various transformations and currently orients itself towards a wide range of related disciplines such as variationist linguistics, comparative linguistics, dialectometry, and natural language processing. An issue of ongoing interest is the definition of the object of study, that is, of the concept of dialect, as well as the classification of varieties or strata within the spectrum between base dialects and standard varieties. As previous research has uncovered many aspects of the structure and classification of traditional dialects, modern studies will keep on focusing on the potential roles, impact, and changes of regionally bound varieties in complex societies and the role of dialects in identity-making.  


\section*{Acknowledgements}
We would like to thank Eleanor Coghill and Elvira Glaser for very useful comments on earlier drafts of this chapter.

\printbibliography[heading=subbibliography,notkeyword=this]
% dois still overhang - hyphenation would solve it?

\end{document}
