\documentclass[output=paper]{langscibook} 
\ChapterDOI{10.5281/zenodo.8269228}

\title{Linguistic accommodation}  

\author{Hanna Ruch\affiliation{University of Zurich} and  Carlota de Benito Moreno\affiliation{University of Zurich}}
 
% \lsConditionalSetupForPaper{} % Please use this instead of loading localpackages.tex, localbibliography.bib, etc. manually

\abstract{This chapter reviews literature on linguistic accommodation and discusses the role of accommodation in language change. In the first part, theoretical models of accommodation and linguistic change are introduced and discussed. In these models, linguistic accommodation (also convergence or synchronization) between individuals is regarded as an important mechanism of language change at the community level. However, more research is needed to validate theoretical models of accommodation and language change. The second part reviews the common research methods of accommodation studies, with a focus on dialect contact. The reviewed studies on short- and long-term accommodation used a large variety of methods and data, which makes comparisons across different studies and languages difficult. The third part of the chapter briefly reviews patterns and processes of accommodation found in the reviewed literature, to identify -- in the fourth part -- the most important linguistic and extralinguistic factors involved in accommodation. The chapter concludes by drawing attention to research gaps in the area of linguistic accommodation and language change, and proposing possible and desired directions for future research.}

\begin{document}
\SetupAffiliations{mark style=none}
\maketitle
\label{chap_accommodation}

\section{Introduction} \label{Section_Introduction}
\largerpage[-1]
We define linguistic accommodation as the adjustments speakers make to become linguistically more (convergence) or less (divergence) similar to an interlocutor, or to a social environment. When they occur in a single interaction or experiment over minutes or hours, we will refer to these adjustments as \textit{short-term accommodation}. \textit{Long-term accommodation} will be used when accommodation takes place over weeks or months, for instance after a speaker has moved to a new region or moved in with a new flatmate.\footnote{Some authors instead use the term ``second dialect acquisition" \citep{siegel_second_2010} and prefer this term over long-term accommodation as it denotes permanent changes \citep{chambers_dialect_1992}. In this chapter, we will nonetheless refer to long-term accommodation, as one of our main aims is a comparison between long- and short-term accommodation.}

As we will see below, accommodation can be observed at all linguistic levels. It can involve the adoption of single elements such as lexical items \citep{brennan_conceptual_1996}, but also more subtle shifts such as a change in speech rate \citep{putman_conception_1984} or degree of regional accent \citep{bourhis_language_1977}. Accommodation can further be observed as categorical switches from one language to another in bilingual speakers \citep{giles_towards_1973}, and is therefore related to language choice and code-switching (see Chapter \ref{chap_codeswitching}).
Given the focus of this book, the present chapter primarily discusses situations involving speakers of different dialects or languages. Nevertheless, research dealing with interlocutors from the same region will be included to shed light on the role of linguistic and extralinguistic factors.


Accommodation can also involve non-verbal communication and other kinds of social behavior \citep{hall_behavioural_2013, dijksterhuis_perception-behavior_2001}. For this reason, the phenomenon has been studied not only in linguistics, but in a range of other disciplines too. The focus of early research on dyadic communication in psychology was primarily on whether speakers converge, for the purposes of understanding interview dynamics \citep[e.g.][]{matarazzo_interviewer_1963}, or investigating the relationship between personality and imitative behavior \citep[e.g.][]{natale_convergence_1975}. In the 1970s, accommodative processes came to the attention of social psychologists who also investigated the role of language and accent in person perception and inter-group processes. 

For linguistics, however, it is crucial to understand \textit{what} linguistic features are subject to the process of accommodation. This issue was soon taken up by sociolinguists, with \textcite{coupland_accommodation_1984} being the first to concentrate on specific linguistic variables. Most of this early sociolinguistic work \citep[e.g.][]{rickford_adressee-_1994,coupland_accommodation_1984,bell_language_1984,selting_levels_1985} aims at understanding style-shifting. \textcite{trudgill_dialects_1986} is probably the first to apply Communication Accommodation Theory (CAT) to dialect contact and dialect change. He formulated the idea that long-term changes in linguistic behavior (i.e. long-term accommodation) are based on repeated short-term accommodation, and further suggests that accommodation between speakers underlies linguistic change at the community level (see below). \textcite{niedzielskietal1996linguistic} propose several ways in which CAT could inform our understanding of language contact phenomena, and encourage linguists to integrate CAT into their research.

Apart from sociolinguistics, accommodation has been examined in other fields of linguistics too. In recent years, the phenomenon has been extensively studied in cognitive psychology and psycholinguistics \parencite{garrodetal_2004_why,casasantoetal_2010_virtually}. This work has mostly used more controlled laboratory settings, and the research aims are mainly oriented toward understanding the mechanisms rather than the social functions of convergence, also referred to as \textit{alignment} or \textit{entrainment}. 
Although many studies on short-term accommodation take place in socially impoverished settings, these experiments have much to say about the linguistic and cognitive factors favoring or inhibiting accommodation. In interactional linguistics, in contrast, speakers' mutual adjustments are of interest to understanding discourse structure and dynamics as well as identity construction through language \citep[e.g.][]{chakrani_arabic_2015,nilsson_dialect_2015}. In these studies, structural patterns are only of secondary interest.

More recently, accommodation has also been studied in applied linguistics and psychology. For instance, accommodation has been used to assess communicative quality in health communication \parencite[see ][]{farzadnia_giles_2015_patient} 
and it is analyzed to improve human-machine interaction  \citep[e.g.][]{linnemann_as_2016}. This work mainly uses holistic or listener-based approaches to quantify accommodation, and does not usually analyze what specific linguistic features speakers accommodate to. In recent years, finally, written computer-mediated communication has also been examined with regard to accommodation \citep[e.g.][]{Danescu-Niculescu-Mizil2011,felder2023individuelle}.

As outlined above, the idea of relating accommodation research to the study of language contact and change is not new. So far, however, there has not been enough empirical research on this issue. The present chapter reviews empirical research on linguistic accommodation, focusing on those aspects which are relevant to the study of language and dialect contact. Therefore, the emphasis will be on studies involving speakers from different dialects or languages. We will start by discussing theoretical models of the relationship between accommodation and contact-induced language change. We will then review the existing literature on accommodation to address the following questions: First, what are the linguistic patterns resulting from short-term and long-term accommodation? Second, what linguistic and extralinguistic factors favor or mitigate accommodation processes? And third, to what extent are these patterns compatible with the idea that contact-induced language change is initiated in individual interactions? We will conclude by proposing directions for future research and by elaborating how accommodation research could further inform our understanding of deep time and societal language contact phenomena (see Chapter \ref{chap_intro}).

\subsection{Linguistic accommodation and contact-induced language change}

Linguistic accommodation has been a crucial element in models of language change. In what follows, we review two of the most relevant proposals of such models, namely, how individual change turns into societal change and how short-term change becomes lasting change. Lastly, we call attention to several other aspects where the role of accommodation in language change is relevant.

\subsubsection{From the individual to the community}

It has long been assumed that linguistic innovations spread via face-to-face contact between individual speakers. As early as the 1930s, \citet[476--477]{bloomfield_language_1933} describes how individual speech habits are shaped by those who the speaker has interacted with before. He also postulates that persons with power and prestige are imitated to a greater extent than socially less influential individuals, and that imitators themselves will become models in later interactions. He further formulates the assumption that, with a few exceptions, ``the process does not rise to the level of discussion''. Moreover, he argues that not all linguistic forms are equally likely to be imitated: ``The adjustments are largely minute and consist in the favoring of speech-forms more often than in the adoption of wholly new ones. A great deal of adjustment probably concerns non-distinctive variants of sound'' \citep[476--477]{bloomfield_language_1933}. Bloomfield thus already describes the general principles of what will later be termed linguistic accommodation, and relates the phenomenon to dialect leveling and linguistic change. 

A more detailed model of the relationship between linguistic accommodation and dialect leveling and change is formulated by \textcite{trudgill_dialects_1986}. He draws a link between social psychologists’ CAT and the question of linguistic diffusion in space, i.e., the micro and the macro level of dialect contact: 

\begin{quote}
Clearly, if a linguistic feature has spread from one region to another, it must have spread from one speaker to another, and then on to other speakers and so on. But how exactly are linguistic forms transmitted from one geographical area to another at the level of the individual speaker \citep[39]{trudgill_dialects_1986}. 
\end{quote}

Trudgill suggests that accommodation is the mechanism of diffusion at the micro level, arguing that ``if a speaker accommodates frequently enough to a particular accent or dialect [. . .] then the accommodation may in time become permanent, particularly if attitudinal factors are favorable'' \citep[39]{trudgill_dialects_1986}. This idea was then taken up by \textcite{auer_role_2005} who refined the so-called change-by-accommodation model. According to their model, short-term shifts may, through repeated interactions, accumulate in long-term accommodation and thus lead to innovation in an individual's speech habit. Given favorable network structures and the critical mass of speakers displaying an innovative feature, the innovation may – again via accommodation – spread to other speakers and lead to linguistic change at the community level. 

\subsubsection{From short-term to long-term}

Although the change-by-accommodation model is widely acknowledged in linguistics, there is a lack of empirical evidence for the idea that repeated short-term accumulates into long-term accommodation.
\textcite{auer_role_2005} compare several sociolinguistic case studies on short-term or long-term accommodation, with linguistic change taking place at the community level. Based on these case studies, they conclude that patterns observed in individual speakers do not align with the change described at the community level. As a result, the authors question the change-by-accommodation model. Their evaluation of the model is mainly based on studies of accommodation which used auditory-phonetic methods. It is thus possible that with more sophisticated, acoustic methods, subtler shifts in pronunciation could be observed, or that other linguistic levels, such as morphology or syntax, behave differently from phonology. 
 
To our knowledge, the only study so far that systematically analyzes variability over short and long time periods is \textcite{sonderegger_medium-term_2017}. This work investigates phonetic variability for five phonological variables (three vowels, stop aspiration and voicing and /t/-deletion) and compares the variability on a daily basis with the variability on a monthly basis in 12 participants of the TV show \textit{UK Big Brother}. The authors' approach permits studying variability – and accommodation – in a closed communication system where the speakers only communicate among themselves, and with nobody from outside the house. They found that day-to-day variability is very common for all speakers and all five variables they looked into. Some speakers showed a trend over time for some variables (i.e. lowering of F2 over several weeks). For many speakers and many variables, however, day-by-day variability did not accumulate into a stable pattern, and overall, there was no evidence for accommodation despite frequent interaction. The only clear evidence for convergence was found for two individuals who also formed a close social bond. \textcite{sonderegger_medium-term_2017} confirm the effect of linguistic as well as by social factors on time-dependent phonetic variability, but they also show that, in their data, short-term trends only occasionally accumulate into longer-term changes. The authors speculate that this is the case because individual speakers exhibit considerable differences in terms of pronunciation plasticity. Based on their findings, \citet{sonderegger_medium-term_2017} speculate that accent change over several years may vary even more  between different speakers, because long-term changes themselves are assumed to build upon medium-term changes. The study suggests that, like short-term accommodation, medium-term dynamics of phonetic variables is mediated by social and linguistic factors as well as individual differences. \textcite{sonderegger_medium-term_2017} relate the important individual differences in phonetic plasticity to the different roles individuals may adopt in the spread of sound change \citep[reminiscent of the contrasts between early adopters and innovators, see][]{milroy_linguistic_1985}.

Further indirect evidence for a more complex relationship between short-term and long-term accommodation comes from studies on long-term accommodation \citep[see][]{ruch_function_2018}. First, most adults hardly ever acquire a second dialect perfectly, even after living in a new social environment for several years \citep{siegel_second_2010}. And second, there are examples of accent reversal, showing that repeated short-term accommodation does not necessarily accumulate over time and therefore does not necessarily lead to long-term accommodation. For instance, the British journalist and radio presenter Alistair Cooke  first converged toward American English after having migrated from the UK to the USA, but shifted back to his British English accent (i.e. reversed his accent) in later life \citep{reubold_dissociating_2015}.  Similar findings are reported by \textcite{werlen_zwischen_2006} who investigate how speakers from Valais, a canton in the southern part of Switzerland, change their pronunciation after relocating to Berne. Two years after relocating, five out of 18 participants used \textit{fewer} Bernese variants than shortly after relocation. More longitudinal studies and more research comparing short- and long-term accommodation within individuals are needed to empirically validate this relationship.

\subsection{Toward an improved change-by-accommodation model}

In this section, we highlight a number of lines of research that have not received as much attention as others, but which seem to us to be of crucial importance to shed light on the role of accommodation in language change. First, it is assumed that long-term accommodation is relevant to understanding contact-induced language change such as, for instance, dialect leveling (\cite{trudgill_dialects_1986}, see also Chapter \ref{chap_dialects}). If a group of speakers moves from region A to region B, this may eventually lead to innovation or contact-induced change in variety B. However, studies on long-term accommodation most commonly focus on mobile speakers, that is, on the effects on variety A, and do not usually address linguistic variability within the receiving community, i.e., effects on variety B.

A possible exception is \citet{klee_andean_2006} who study the speech of Andean migrants in Lima and also analyze a control group of lower-class Limeños, the social group most likely to be in contact with the migrant population. \textcite{klee_andean_2006} find no evidence of change within the receiving community's variety as a result of contact with migrants. That is, this study does not support the idea that migrants spread linguistic features to a new community. \textcite{Escobar2007}, on the contrary, suggests that migrant speakers brought Andean Spanish features into \textit{costeño} Spanish spoken in Lima, although she considers this influence to be restricted to syntactic features of low sociolinguistic salience. Ideally, future work would concentrate not only on mobile individuals, but also investigate the possible effects on the variety of the receiving community. We argue that in order to understand contact-induced change, the receiving community is as important as the migrating individuals. Long-term changes in the speech of mobile individuals, on the other hand, provide ideal scenarios for studying dialect attrition within individuals.

Second, although most research on accommodation has dealt with adult speakers, children may be as relevant as adults when it comes to testing and refining the change-by-ac\-com\-mo\-dation model. It is generally acknowledged that children acquire a second dialect more quickly and more easily than adults \citep{siegel_second_2010} when moving to a new environment. At the same time, they seem to be quite sensitive to linguistic variation from early on. For instance, \textcite{jones_development_2017} show that even some of the 4--5-year old participants are able to distinguish their own regional variety from other varieties of American English. \textcite{khattab_phonetic_2013} describes how three children between 5 and 10 years of age converge and diverge in the use of local, standard and non-native phonetic features in English when interacting with their mothers. 
Children might be relevant to dialect leveling and change for several reasons. They may acquire a dialect imperfectly, bringing D1 features into D2, but may also become bidialectal speakers, that is, become fluent in both dialects while still separating them. For instance, they may use D1 at home, and D2 in school and elsewhere. Finally, children may also end up with a mixed variety \citep{chambers_dialect_1992,tagliamonte_howd_2007}, which \textcite{klee_andean_2006} regard as a possible source for dialect leveling and change, presupposing a critical mass of speakers.

Third, the model remains rather vague about how exactly contact between speakers takes place, and about the kinds of situations that facilitate either short- or long-term accommodation. A central question is whether the former or the latter has more impact on a given linguistic variety. More concrete predictions and, ideally, their empirical validation would allow for linking these ideas to issues of areal linguistics (see Chapters \ref{chap_dialects} and \ref{chap_areas}). We can think of at least two scenarios leading to the patterns found in areal linguistics. First, speakers are more likely to move to close-by, culturally and linguistically similar areas \citep[e.g.][]{falck_cultural_2016}. In this case, linguistic similarity would be induced by the mobile speakers' influence on the local dialect. Alternatively, places within shorter travel distances might favor frequent short-term contacts, for instance, through trading, commuting, etc. In the latter case, dialect change and leveling would take place through repeated short-term accommodation in face-to-face interactions.

\section{Approaches}

In this section, we present the most common methods used in accommodation research. We will start by presenting the methodological approaches to short-term accommodation and then discuss the most common methods that have been used to study long-term accommodation.

\subsection{Short-term accommodation}
Studies on short-term accommodation can roughly be divided into two types: dialogue studies and shadowing tasks. Dialogue studies analyze recorded dialogues between speakers, mostly between unacquainted persons. In most study designs, the participants are given a collaborative task such as describing a route on a map to their interactant \citep[i.e. a map task, e.g.][]{pardo_phonetic_2006}, or finding the differences on otherwise identical pictures \citep[i.e. a diapix task, e.g.][]{kim_phonetic_2013}. In other work, participants are asked to converse freely \citep[e.g.][]{schweitzer_convergence_2013}. Dialogue studies represent more natural speech situations than shadowing tasks, making them suitable to investigate socio-psychological issues such as the relationship between accommodation and speaker perception.

So-called shadowing tasks \citep{goldinger_echoes_1998,shockley_imitation_2004,babel_dialect_2010}, in contrast, involve more controlled situations, which makes them particularly appealing for studying the effect of linguistic factors. The experiments typically comprise three phases: (1) recording the participants' baseline productions, (2) having participants listen to the speech of a model speaker over headphones and (3) recording the participants' post-task speech. Post-task productions are then compared to the baseline productions to see whether the participants became linguistically more similar, that is, whether they converged towards the model speaker. Variations of the paradigm have been implemented in web-based experiments \citep{weatherholtz_socially-mediated_2014} and in experiments involving nonhuman model speakers \citep[e.g.][]{beckner_participants_2016}. The listening task can consist of isolated words \citep[e.g.][]{goldinger_echoes_1998} or a longer passage \citep[e.g.][]{yu_phonetic_2013,weatherholtz_socially-mediated_2014}. Sometimes, listeners are asked to repeat each word separately, while in other cases, the listening and speaking tasks are taken in blocks, implying a longer pause between the listening task and the post-task production.


The methods to assess accommodation also vary considerably across studies and subdisciplines. Dialogue studies have often assessed accommodation by asking independent listeners to judge the similarity of dialogue excerpts \citep{pardo_phonetic_2006,kim_phonetic_2013}. This approach has been used in several shadowing tasks too \citep[e.g.][]{goldinger_echoes_1998}, turning out to be a very useful method for assessing the global similarity of isolated words. In other phonetically-oriented studies, specific parameters are measured \citep[e.g.][]{babel_dialect_2010,de_looze_investigating_2014} which, however, correlated only marginally  with perceived similarity as assessed by independent listeners \citep{pardo_phonetic_2013,walker_repeat_2015,abel_cognitive_2016,pardo_phonetic_2017}. Research on lexical, syntactic, or morphological accommodation usually quantifies the frequency of the linguistic variants under study \citep[e.g.][]{beckner_participants_2016,weatherholtz_socially-mediated_2014}.

These differences in research design as well as in the quantification of accommodation make comparisons across studies difficult. For these reasons, in Section \ref{Section_Factors}, rather than compare the degree of accommodation or other details across studies, we will organize the findings of accommodation according to the research questions outlined in  Section \ref{Section_Introduction}: What are the linguistic patterns resulting from short- and long-term accommodation? What linguistic and extralinguistic factors favor or mitigate accommodation processes?

\subsection{Long-term accommodation}
\begin{sloppypar}
Studies on long-term accommodation typically focus on speakers who have moved from their region of origin to a place where a linguistic variety different from their own is spoken. Studies on long-term accommodation are frequently framed within a sociolinguistic approach. This means that they typically rely on semi-spontaneous speech, often collected by means of sociolinguistic interviews (\cite{shockey_all_1984, auer_subjective_1998, romera_prosodic_2013}, among many others). In longitudinal studies, the same speakers are recorded several times after having moved to a new region, which allows tracking an individual's linguistic shifts over time. Probably because of the considerable logistic effort needed, longitudinal studies are rather rare \citep[but see][]{shockey_all_1984,auer_subjective_1998,reubold_dissociating_2015}.
\end{sloppypar}

An exception to this, however, are studies on the effect of accommodation on children and youngsters. These are often longitudinal. For instance, \citet{chambers_dialect_1992} records his speakers twice in a two-year period, while \citet{tagliamonte_howd_2007} record their participants every weekend starting six months after having moved from Canada to England. The often large time lapses between interviews are due to logistic challenges. In \citet{tagliamonte_howd_2007}, however, the subjects are the first author's children, a fact that facilitated data collection. At any rate, the majority of investigations concerned with long-term accommodation rely on data collected once for each subject. Usually, the participants' speech after migrating is then compared to existing, general descriptions of their linguistic variety \citep{shockey_all_1984, trudgill_dialects_1986, MolinaMartos2010}, or to non-mobile speakers from their place of origin \citep{PalaciosAlcaine2007,Fernandez2013}. To investigate the effect of time of exposure on accommodation, time spent in the new environment is usually used as a predictor \citep{shockey_all_1984, romera_prosodic_2013, erker_contact_2016}, although this parameter of course does not necessarily correlate with the actual amount of linguistic exposure to the new variety. In comparison to short-term studies, which often follow a controlled, experimental protocol, longer-term changes in speech are much more difficult to trace back to specific factors. Some studies have used questionnaires in order to gain additional information about the speakers' social environment or attitudes \citep[e.g.][]{pesqueira_cambio_2008}.


\section{Patterns and processes}

It is useful to distinguish between patterns of accommodation, i.e. its possible outcomes, and the processes whereby accommodation takes place. We discuss both in what follows.

\subsection{Patterns}

\textcite{giles_accommodation_1991} distinguish between three  accommodative patterns: convergence, divergence and maintenance. Convergence describes the situation where speakers become more similar to their dialogue partner or a model speaker. In divergence, individuals become more dissimilar to their conversation partner or to a model speaker. Maintenance, finally, denominates the case where an individual does not shift toward or away from another speaker, but largely maintains their way of speaking. In dialogues, convergence and divergence can be reciprocal, but also asymmetric in the sense that one, but not the other speaker, converges or diverges. \textcite{giles_accommodation_1991} further note that speakers may converge on some parameters, while diverging on others.

As is apparent from the present chapter and from previous work reviewing accommodation studies \citep{ruch_function_2018}, convergence seems to occur much more frequently than divergence. One possible explanation for this bias is that alignment is the default pattern and, as a consequence, is observed much more frequently than maintenance or divergence \citep[see][]{dijksterhuis_perception-behavior_2001}. However, another possible explanation is that, given that convergence is the expected result, divergence is not as thoroughly scrutinized by researchers. It may also simply be that null results or divergence are more difficult to publish. This could have led to a publication bias toward convergence. For syntactic accommodation, divergence indeed seems to receive some support in the literature. In order to actively engage with their interlocutor, speakers seem to use complementary structures rather than repetition \citep{healey_divergence_2014}.

Given that analysis and quantification of accommodation differ considerably across studies, it is extremely difficult to describe linguistic patterns in accommodation more generally. As mentioned above, socio-psychological work so far has mainly focused on whether accommodation was observable and has therefore used perceptual, more holistic measures of accommodation. Work within computational linguistics, too, has used holistic measures \citep{lewandowski_talent_2012,de_looze_investigating_2014}, however, often without relating them to linguistically interpretable categories. More recent work within linguistics and psycholinguistics has focused on a limited number of specific linguistic features. The features in these studies mostly belong to a single level of linguistic description only, for instance, voiceless stops \citep{nielsen_specificity_2011}, vowel quality \citep{babel_dialect_2010}, past tense formation \citep{beckner_participants_2016}, or the English dative alternation \citep{weatherholtz_socially-mediated_2014}. Most studies involving dialect contact deal with phonetics or phonology. This is the case, perhaps, because in this area, dialectal differences are most obvious and better described than, for instance, in morphology, syntax, or pragmatics. Furthermore, when working with spontaneous or semi-spontaneous speech, it is more feasible to get a sufficient number of tokens for phonetic or phonological features than, for instance, for syntax or lexis.

For these reasons, in Section \ref{Section_Factors} we will refrain from listing different linguistic phenomena observed in accommodation research. Instead we will group and discuss the observed patterns according to the linguistic and extralinguistic factors that have been shown to favor or inhibit accommodation.

\subsection{Processes}

The two most influential models dealing with the processes underlying accommodation are the Communication Accommodation Theory (CAT) and the Interactive Alignment Model (IAM).
CAT \citep{giles_accent_1973,giles_towards_1973,giles_speech_1975} was developed in the field of social psychology and primarily attributes a social function to accommodation. Convergence and divergence are seen as the speakers' communicative strategies to express social closeness or social distance in an interaction \citep[293]{giles_communication_2007}. The model thus focuses more on the ultimate function of accommodation, rather than on its underlying mechanisms.

IAM \citep{pickering_toward_2004} has its origins in cognitive psychology and sees convergence as an automatic process, which results from a link between speech perception and speech production. This link is similar to the priming mechanism and is constantly activated during speech processing \citep{pickering_toward_2004}. In some cases, it is difficult to separate accommodation from priming. We follow \textcite{pickering_toward_2004} who regard priming as the underlying mechanism of accommodation, whereas accommodation is the process of mutual linguistic adjustments in its communicative context.

At first sight, the two models might seem conflicting, because a phenomenon which results from an automatic process is not necessarily assumed to have a social function. However, the two models can also be seen as complementary and, as is for instance common practice in biology \parencite{tinbergen1963aims}, mechanism and function can be studied independently from each other \citep{ruch_function_2018}.

\section{Factors} \label{Section_Factors}

We will now discuss the findings from the accommodation literature with respect to evidence for linguistic and extralinguistic factors. As much as possible, findings from long-term studies will be compared with those from short-term studies to explore the extent to which short- and long-term accommodation could potentially be based on the same mechanisms and governed by similar constraints.

\subsection{Linguistic factors}
From a linguistic point of view, accommodation studies seek to answer two important questions. First, what kind of linguistic features are more susceptible to convergence, and second, what factors favor or inhibit this process? A number of studies have highlighted the role of \textit{salience} in long-term accommodation. Salience can be defined as perceptual conspicuousness of a linguistic element \citep{lenz_zum_2010}. Since it arises in context, it cannot be defined in absolute terms.  Salience of a linguistic element is assumed to be affected by acoustic, cognitive and sociolinguistic factors \citep{auer_anmerkungen_2014}.\footnote{Other work has used subjective criteria to operationalize salience \citep[see examples reviewed in][]{wilson_types_2011,macleod_critical_2015}. Criteria based on the researcher's perspective, however, are problematic because they impede comparisons across studies, and because salience as perceived by language users themselves, rather than by the researcher, is arguably more relevant \citep[see][]{macleod_critical_2015}. See Section \ref{Section_Discussion} for further argumentation and examples.}

Several studies report more convergence toward a second dialect for salient features of the D2 \citep{auer_subjective_1998,pesqueira_cambio_2008,wilson_types_2011,romera_prosodic_2013}. That is, salient features of a variety seem to be more easily picked up by D1 speakers. 
However, convergence for salient features does not always occur and seems to be mediated by social attitudes. For instance, it has been noted that while D2 stereotypes are rarely adopted (sometimes they are even diverged from), D1 stereotypes are easily abandoned and, consequently, result more easily in convergence \citep{trudgill_dialects_1986,erker_contact_2016}. \citeauthor{Escobar2007}'s (\citeyear{Escobar2007}) finding that only syntactic features with low salience were transferred from (highly stigmatized) Andean Peruvian Spanish to \textit{costeño} Peruvian Spanish, points in the same direction. Research on short-term accommodation is generally consistent with these findings, suggesting that some linguistic features are more easily adopted than others \citep{babel_dialect_2010,walker_repeat_2015}. \citet{babel_dialect_2010} argues that New Zealanders possibly converge less toward the Australian KIT and TRAP vowels (/ɘ/ and /ɛ/ in New Zealand, /ɪ̠/ and /æ/ in Australian English) because these are particularly salient Australian features from the perspective of New Zealanders. Similar arguments can be found in \citet{walker_repeat_2015} for the variable imitation of different vowels across varieties of English. However, in none of these publications is salience quantified empirically, and thus the findings remain speculative. A possible exception is \textcite{macleod_effect_2012}, a study that explicitly investigated the role of perceptual salience on short-term accommodation. Salience is assessed here by means of a dialect recognition test. Features contributing more to dialect recognition are considered to be more salient. Interestingly, perceptual salience is able to predict the degree but not the direction of accommodation. This  seems to depend, instead, on the participants' attitudes toward the interlocutor's dialect and toward the new social environment.

Another important factor seems to be \textit{intelligibility}. D1 phonetic features that frequently cause misunderstandings with D2 speakers are more susceptible to accommodation \citep{trudgill_dialects_1986}. \citet{shockey_all_1984}, for instance, observes a greater decrease of /t/-flapping than /d/-flapping in speakers of American English who have moved to Britain. This result might be explained by the low frequency of /t/-flapping, but not /d/-flapping, in British English. Given that /t/-flapping potentially leads to misunderstandings in British English, American speakers seem to accommodate more easily toward British English for this variable. Similarly, the fact that lexical differences are highly salient and can cause severe and obvious comprehension difficulties \citep{trudgill_dialects_1986} might explain why the lexicon is usually the first linguistic level to be affected by accommodation \citep{Bonomi2010,chambers_dialect_1992}.
Results from short-term studies are generally consistent with these findings. In a dialogue study, \textcite{hwang_phonetic_2015} find that non-native speakers of English pronounce plosive and vowel contrasts in a more English-like way in words with a phonological competitor. They interpret this result as evidence for accommodation to the pragmatic needs of the listener.
A seminal study on functional constraints in short-term accommodation was conducted by \citet{nielsen_specificity_2011}. She tested the effect of lengthened and shortened voice onset time (VOT; i.e. amount of aspiration or voicing of a plosive) in /p/ on its imitation. Interestingly, participants imitated lengthened, but not shortened VOT. This result is interpreted with the phonological status of VOT in English. While lengthening VOT (i.e. aspiration) does not have phonological consequences, VOT shortening may lead to a confusion of /p/ with /b/ in minimal pairs such as \textit{pan} versus \textit{ban}.

Yet another linguistic variable that favors imitation is \textit{linguistic variability}. In a comparison between mobile and non-mobile adult speakers of American English, \citet{bowie_effect_2000} finds that, in the long term, phonological variables that are currently undergoing linguistic change are more susceptible to adaptation than more stable features. As for short-term accommodation, \citet{watt_levels_2010} observe that an interviewer in the Scottish-English border region is more inclined to converge toward their interviewees for variable than for stable linguistic features. Similar results come from one of the few studies exploring morphological convergence. Using an adapted version of Asch's conformity experiment (1951), \citet{beckner_participants_2016} test whether human participants are influenced by human or robotic peers in their way of forming the English simple past. The participants' morphology is influenced by humans, but not robots. In verbs with variable past tense formation (e.g. \textit{dream - dreamt/dreamed}) the subjects are more likely to imitate the human peer's choice.
It has also been claimed that free variation  (i.e. altering the pronunciation of one phoneme in every context) is more prone to accommodation than conditioned variation (where the pronunciation of a sound is affected only in some contexts) \citep{trudgill_dialects_1986,siegel_second_2010}. \textcite{chambers_dialect_1992} rephrases this constraint by distinguishing between simple and complex phonological rules. Simple rules (such as /t/-voicing in English) are categorical in the sense that they have no exceptions, while complex rules (such as vowel backing in English) do not automatically apply in all contexts. In his study of anglophone Canadian youngsters in the south of England, he finds that Canadian /t/-voicing is abandoned faster (implying convergence toward British English) than the British process of vowel backing is acquired. \textcite{wilson_types_2011} finds similar results for speakers of Moravian who had moved to Prague and converged to Common Czech, although he notes that rules are seldom without exception and prefers to use the term ``semi-simple rules.''

There is also evidence that accommodation is affected by lexical factors. For instance, for Argentinians who had moved to Mexico City, \citet{pesqueira_cambio_2008} finds more phonetic accommodation in highly frequent words. This result can be explained by the enhanced degree of exposure for these items. However, in some short-term studies, shadowers are found to converge \textit{less} toward their model speakers with respect to high-frequency words \parencite{goldinger_echoes_1998,goldinger_episodic_2004,babel_dialect_2010,nielsen_specificity_2011}. This apparent contradiction between short- and long-term studies can be resolved by considering high-frequency words in long-term studies as words that are repeated more often and, therefore, provide the speakers with a higher degree of exposure to these words. The results from short-term studies, in contrast, have been explained by the episodic traces left by the tokens heard, which are assumed to be less influential in high- compared to low-frequency words \citep{goldinger_echoes_1998}, an interpretation that is in line with Exemplar Theory \citep{pierrehumbert_exemplar_2001}. However, a recent comprehensive study on short-term accommodation \citep{pardo_phonetic_2017} was not able to replicate the main effects of frequency found in earlier work, but instead found an interaction between speaker gender and word frequency (see below).

In long-term studies, D1 phonetic features have been found to be more likely to persist in words where these features were lexicalized \citep{auer_subjective_1998}, or in forms which do not exist in D2 at all \citep{pesqueira_cambio_2008}. Similarly, words that exclusively exist in D2 seem to facilitate the adoption of D2 phonetic features \citep{pesqueira_cambio_2008}. In line with these results, \citet{Bonomi2010} observes that discursive markers and words related to the new cultural reality are adopted first by Spanish-speaking individuals who have migrated from Latin America to Spain and Italy.

In order to become a relevant force in language change, accommodation not only must show some consistency across speakers, but should also generalize across the lexicon and across different syntactic constructions.
Some evidence for \textit{generalizability} comes from short-term studies. For instance, in her shadowing task, \textcite{nielsen_specificity_2011} finds that speakers of American English not only imitate lengthened VOT in items with word-initial /p/, but also generalize this sub-phonemic specificity to new instances of /p/ and even words with initial /k/. \textcite{beckner_participants_2016} find that some of their participants generalized the  morphological pattern heard from the model speaker (regular past tense formation in English) to new verbs.

There is some disagreement on how \textit{linguistic distance} between the systems in contact influences accommodation. \textcite{kim_phonetic_2011} find more convergence between speaker pairs of American English who are from largely the same dialect region than between speaker pairs from different dialect regions. \textcite{ruch2021dialect} found no convergence between speakers from two different regions of Switzerland after they were exposed to each other's speech in a dialogue. 
In contrast, \textcite{babel_evidence_2012} finds the most convergence for exactly those vowels and participants who differ most from the model speaker. Large phonetic distance between the participants and the model speaker also favor phonetic convergence in a study by \textcite{walker_repeat_2015}. The findings mentioned above \parencites{bowie_effect_2000}{watt_levels_2010}{beckner_participants_2016}, that synchronic intra-speaker variability favors convergence, offer yet another interpretation: speakers will more readily take up and use a variant that is a plausible token of their own distribution for the same linguistic variable \citep[for evidence from an agent-based model, see][]{harrington_/u/-fronting_2017}. 

While the focus of this chapter is on dialect contact, it is worth mentioning that accommodation has also been found to occur between bilingual speakers with varying degrees of L2 proficiency. Over longer time periods, the predominant linguistic environment has been shown to not only affect a speaker's L2, but also her L1. For instance, in a bilingual speaker of Portuguese and English, VOT is longer or shorter after a stay of several months in Brazil or the USA, respectively \citep{sancier_gestural_1997}. \citet{tobin_phonetic_2017} partly replicate these findings for a larger set of Spanish-English bilinguals with Spanish as a dominant language. The speakers' VOT in English voiceless stops drifts toward that of the ambient language (Spanish or English), however, no drift is observed for VOT in Spanish, which is the speakers' L1. \textcite{chang_rapid_2012} studies American English learners of Korean and finds that already after a few weeks in Korea with intensive Korean classes, the English speakers' L1 is phonetically influenced by the L2. In a subsequent study, \textcite{chang_novelty_2013} shows that the phonetic drift toward L2 is less pronounced in more experienced learners.

An interesting aspect of these findings is that the ambient language not only affects the language currently heard and spoken by the speakers, but also their other, ``inactive'' language. These effects on the L1 are often considered cases of linguistic attrition (see Chapter \ref{chap_shift} for a more general discussion of attrition and shift) and have been shown to affect all linguistic levels, including morphosyntax. \citet{kaufman_morphological_1991}, for instance, analyze the effect of English on Hebrew in a two-year-old after moving from Israel to the US. Their longitudinal study shows how Hebrew inflectional and derivational morphology are simplified, resulting in an idiosyncratic mixed variety \citep{kaufman_morphological_1991}.

Short-term studies involving conversations between L2 and L1 speakers are to some extent compatible with these findings. \textcite{lewandowski_talent_2012} finds mutual phonetic convergence between German speakers and native speakers of English in English conversations. Interestingly, native English speakers converge, even though prior to the dialogue they have been instructed not to do so.
In contrast, \textcite{kim_phonetic_2011} find convergence for some pairs and divergence for others, between native and non-native interlocutors of English. The authors argue that the heavily-accented L2 English of most of their non-native speakers might have enhanced the processing load and therefore inhibited convergence \parencite{kim_phonetic_2011}. \textcite{berry_phonetic_2017} analyze two vocalic contrasts in Spanish and Dutch speakers of English. Prior to the dialogue, Spaniards produce the /ɛ/-/æ/, but not the /i/-/ɪ/ contrast, while Dutch participants produce the latter, but not the former phonological contrast. During a conversation in English with a Dutch native speaker, Spaniards converge toward their Dutch confederate by merging /ɛ/-/æ/ and unmerging /i/-/ɪ/.

Taken together, these results suggest that not only categories in an L2 but also in an L1 are more malleable than previously thought. \textcite{hwang_phonetic_2015} analyze two phonological contrasts in conversations between Korean speakers of English in a separate collaborative task with (a) a native speaker of English and (b) a partner who speaks English with a heavy Korean accent. Participants converge toward the English native speaker, but only \textit{after} the latter has produced the phonological contrasts of interest. No convergence toward the Korean confederate is observed, however. Based on their results, the authors conclude that accommodation is better explained as as result of priming, not as a way of affiliating with the conversation partner. \textcite{kootstra_syntactic_2010} find similar results for Dutch-English bilinguals in situations with code-switching. In an experimental setting, they find that the utterances of the confederate have an effect on the speakers' word order in both their L1 and their L2. While \textcite{kootstra_syntactic_2010} interpret their results with the Interactive Alignment Model, they could also be interpreted in terms of CAT (i.e. convergence as an attempt to affiliate with the interlocutor) or in terms of priming.

\subsection{Extralinguistic factors} \label{subsec_extralinguistic}
A common finding of most research on accommodation is that there are important differences between individual speakers in the extent, and sometimes also the direction, of accommodation \citep[e.g. ][]{yu_origins_2013,macleod_effect_2012,babel_evidence_2012,werlen_zwischen_2006,evans_plasticity_2007}. In some cases, these individual differences can be traced back to individual differences in, for instance, attitudes, personality, or exposure to a new linguistic environment. In other cases, interaction-related variables can explain at least some of the variability. In what follows, we will again compare findings from long-term studies against results from research on short-term accommodation where this is possible. There is some evidence for the role of speaker age in accommodation. When exposed to a new linguistic environment for a longer time period, children acquire a new dialect faster than adults and, in some cases, they acquire it almost completely \citep{chambers_dialect_1992,siegel_second_2010,tagliamonte_howd_2007}.  \citet{chambers_dialect_1992} distinguishes between early and late acquirers. Children younger than seven are typically early acquirers and reach native-like levels in the second dialect, while adolescents older than 14 are typically late acquirers and will not completely acquire the second dialect. In fact, many studies highlight that, similar to second language acquisition, adolescents and adults hardly ever master second dialects \citep{siegel_second_2010}. For his sample of 39 Moravians living in Prague, \textcite{wilson_types_2011} reports on only two subjects who acquired native-like levels for the phonetic and morphological variables studied. A large majority (36 out of 39) of the participants accommodates to variable extents and one speaker does not accommodate at all, maintaining their native dialect.

These findings are consistent with the differences found between first and second generation migrants in \textcite{klee_andean_2006}: While Andean migrants who have moved to Lima maintain many of their Andean Spanish features, their Lima-born children are almost indistinguishable from other Limeños \parencite[ the linguistic effect of having non-native parents, see][]{payne_factors_1980}. Another example for imperfect acquisition comes from intermediate forms. Sometimes, D1 variants change toward intermediate variants between D1 and D2 (so-called interdialect forms). For instance, \citet{PalaciosAlcaine2007} observes that, after having moved to Madrid, adolescents from Ecuador tend to both abandon the evidential values of their native compound past tenses and to use these tenses more often, as typical for Madrid speech. However, their use still differs from that of Madrid speakers and thus represents a mixed use.

In some long-term studies, hyperdialectalisms are observed, which can be interpreted as a result of overgeneralization \citep{trudgill_dialects_1986}.
\textcite{klee_andean_2006}, for instance, find that some Andean migrants show higher frequencies of /s/-aspiration and /s/-elision than native Limeños. In line with these results, migrants are commonly perceived to neither speak D1, nor D2 \citep{siegel_second_2010}, but an intermediate or mixed dialect.
Very few studies so far have been concerned with the relationship between age and short-term accommodation. In line  with the age-effects reported for long-term accommodation, \citet{nielsen_phonetic_2014} finds that in a shadowing task, children imitate lengthened VOT to a greater extent than adults. However, more research is needed to understand how short-term accommodation evolves across the life-span and, in particular, in childhood. 

In the sociolinguistic literature,  speaker gender and its relation to linguistic variation has been extensively studied. Women have often been ascribed a crucial role in language change \citep{labov_intersection_1990}, and some long-term studies suggest that women are more prone to converge to a new variety than men. For instance, Argentinean women use a higher percentage of Mexican Spanish phonetic forms than men after residing for several years in Mexico City. \citet{pesqueira_cambio_2008} and \citet{MolinaMartos2010} observe that female Latin-American immigrants in Madrid use more European Spanish courtesy forms than men. In the latter study, however, women also show more negative attitudes toward Madrid speech than their male compatriots. This finding is interpreted as a sign of women attempting to improve their social status by converging toward the local norms.

Gender differences in accommodative behavior have been interpreted in various ways. For instance, \citet[20--21]{giles_accommodation_1991} look at them in the context of social power relations, similar to the situation that salespersons converge more to their clients than vice-versa. \textcite{chambers_dialectology_1998} hypothesize that women, perhaps as a result of fewer opportunities for occupational achievement (still relevant today), tend to fulfill a higher number of different social roles than men. As a result, women come into contact with more people within more different social environments, and therefore ``must master a wider repertoire of linguistic variants than men'' \citep[85]{chambers_dialectology_1998}. \citet{willemyns_accent_1997} suggest that gender differences in accommodative behavior may be related to women being more affective than men, and \citet{namy_gender_2002} relate these differences with gender-related differences in sensitivity to indexical variation, that is, systematic linguistic variation associated with extralinguistic factors such as the social background of the speaker or the social context in which the communication takes place. \citet{namy_gender_2002}  assume that differences in sensitivity to indexical variation might themselves be related to social or affiliative motives.

\textcite{tagliamonte_howd_2007} also observe gender differences in the acquisition of the British English glottal stop by Canadian youngsters. They also note, however, that these differences parallel the sociolinguistic distribution of the variants in the native population. Rather than seeing an effect of the child's gender, they see their results as an example for how children acquire socio-indexical variation. Two recent, very comprehensive studies \parencite{pardo_phonetic_2017,pardo_comparison_2018}, in contrast, are not able to replicate the gender effects reported in earlier studies. Overall, no differences in degree of convergence are observed between women and men. Interestingly, however, women appear to be slightly more sensitive to factors influencing convergence: In \textcite{pardo_phonetic_2017}, speaker gender interacts with lexical frequency, with women being more prone to imitate model speakers in low-frequency words. The authors suspect that gender effects in earlier shadowing tasks might be driven by the use of low-frequency words in some studies or by individual model speakers. In \textcite{pardo_comparison_2018}, which assesses convergence in both shadowing tasks and conversations, women's accommodative behavior is less consistent across tasks than men's. Again, this result suggests that women are more sensitive to factors that seem to mediate linguistic accommodation.

One of the most relevant factors to explain individual variability are speakers' attitudes. Speakers with more favorable attitudes toward a new variety and the receiving community (measured as, for instance, the speakers' willingness to stay or their plans to return) have been found to accommodate to a greater extent than those with less positive attitudes in several long-term studies \citep{VandenBerg1988,werlen_zwischen_2006,pesqueira_cambio_2008,romera_prosodic_2013,Mick2013,reubold_dissociating_2015}. Hence attitudes toward one's own and the new linguistic variety seem to play a crucial role in long-term accommodation \citep[see][]{Caravedo2010}. In the first place, they may affect an individual's willingness to integrate in the receiving community and, in addition, these attitudes seem to be related to establishing new social relationships.

Studies on short-term accommodation found comparable results for the role of speakers' attitudes. \citet{macleod_effect_2012} observes that Argentinian speakers with plans to stay in Madrid are more likely to converge toward a Madrid speaker than those with less-positive attitudes toward their new social environment. However, in this study short-term effects are not easily separable from long-term effects, because at the time of the study, the participants had been living in Madrid for different lengths of time. 
Similarly, more positive attitudes toward the interlocutor lead to more convergence in a number of other studies \citep{babel_dialect_2010,babel_evidence_2012,yu_phonetic_2013,schweitzer_convergence_2013}, or to less divergence in a few others \citep[e.g.][]{schweitzer_social_2014}. 


\subsubsection{Interaction-related factors}

Some effects on accommodation have been shown to depend neither on linguistic, nor on speaker-specific factors, but may be better explained by the specific situation in which an interaction takes place. 
For instance, the way a model speaker is presented (either positively or negatively) affects the extent to which participants imitate the model speaker's long VOT in a shadowing task \citep{yu_phonetic_2013}. In an earlier study, however, a similar manipulation did not affect the participants' degree of accommodation \citep{babel_dialect_2010}. The findings mentioned above are generally compatible with long-term studies showing that  positive attitudes toward the new social environment facilitate convergence toward the new linguistic variety \citep{werlen_zwischen_2006,macleod_effect_2012,pardo_phonetic_2012}.

The only investigation so far which directly compares accommodation in shadowing tasks and unguided interactions \parencite{pardo_comparison_2018} finds that the degree of convergence (as assessed by independent listeners in a perception task) is very similar across tasks. Overall, degree of convergence between speakers is not correlated across tasks. A weak correlation between degree of convergence in the two types of tasks is found for male, but not for female participants. This finding is important because it suggests that results from non-interactive tasks cannot easily be generalized to speech in more natural, interactive settings \citep{pardo_comparison_2018}. 

Research on dialogues by \textcite{pardo_phonetic_2006} and \textcite{pardo_phonetic_2013} shows that the specific communicative role an interlocutor has in a conversation can also affect accommodation. If convergence was based on exposure alone, we would expect less active dialogue partners to converge to a lesser degree than participants who speak more. However, \textcite{pardo_phonetic_2006} and \textcite{pardo_phonetic_2013} find that for vowel quality and speech rate, information givers converge more toward information receivers than vice versa. \textcite{pardo_phonetic_2013} explain their findings in terms of social affiliation. Speakers who are more interested in information transfer (i.e. the information givers), are more inclined to affiliate with their dialogue partners and therefore converge more.

A number of phonetic studies suggest that convergence is contingent on cognitive load. \textcite{abel_cognitive_2016} find that speakers converge only in a simple, but not in a difficult collaborative task. \textcite{berry_phonetic_2017} find more convergence of Spaniards toward Dutch speakers of English in an informal than in a formal situation. Furthermore, convergence is positively correlated with a participant's proficiency in English in this study. These findings suggest that in a situation with lower processing costs, speakers pay more attention to their interaction partner's speech, and therefore are more likely to converge \citep{yu_phonetic_2013,abel_phonetic_2011,berry_phonetic_2017}.

\section{Discussion and outlook} \label{Section_Discussion}

The main aim of this chapter has been to compare short- and long-term accommodation and to discuss their relevance to the change-by-accommodation model. We will start by summarizing our findings and then move on to formulating new research questions and highlighting promising areas for future research.
Among the linguistic factors mediating accommodation, both intelligibility and linguistic variability show consistent results between long- and short-term studies. According to the reviewed literature, linguistic features that impede intelligibility as well as features that exhibit synchronic variation are accommodated faster than other linguistic features. The effect of lexical factors such as word frequency appeared to differ between long-term and short-term studies. While low-frequency words facilitate convergence in short-term studies, long-term studies find that more frequent words were more prone to converge. As stated earlier in this chapter (see Section \ref{Section_Factors}), this apparent contradiction can be resolved by considering the degree of exposure. 

While salience is one of the most-studied factors in the accommodation literature, the many different approaches to the concept prevent a direct comparison between different studies, both across and within long- and short-term accommodation. \textcite{auer_anmerkungen_2014} distinguishes three types of criteria that contribute to the perceptual salience of a linguistic feature: acoustic-auditory factors, cognitive factors and sociolinguistic factors. Given that these factors are not independent from each other (e.g. a longer, acoustically salient vowel is more prone to acquire sociolinguistic salience), different aspects of salience are hard – if not impossible – to operationalize.

It seems to us that a more fruitful approach to the study of salience would entail a listener-based approach \citep{macleod_critical_2015,ruch_role_2018}. Instead of estimating salience based on theoretical criteria from a researcher's perspective \citep{auer_subjective_1998,trudgill_dialects_1986}, listener-based approaches work with experiments or questionnaires. For instance, \textcite{ruch_role_2018} uses a perception experiment to operationalize the salience of phonetic features in two Swiss German dialects. Native listeners of Grison and Zurich German were asked to identify the dialect of spoken isolated words which contained different segmental cues to one of the two dialects. By measuring sensitivity and reaction time it is possible to rank the different segments according to their salience. \textcite{ruch_role_2018} finds that the most salient dialect features are also the ones people from all over German-speaking Switzerland most frequently mention when asked to describe the dialects in an online questionnaire. This suggests that a first and feasible approach to learn about salient features of a variety is by asking (naive) listeners to describe  how they recognize speakers of the variety in question.

As discussed in Section \ref{subsec_extralinguistic}, among the extra-linguistic factors, attitudes and age show the most consistent effects between long- and short-term studies. More positive attitudes toward the contact variety and a younger age seem to facilitate convergence toward a different dialect. However, more research is needed on the speech of children and adolescents, for whom short-term accommodation is still under-researched.

The role of gender, in contrast, is controversial in accommodation. Some studies find that women converge more than men, in both the short and the long term. However, such gender differences in accommodative behavior surface only in few studies. Furthermore, recent research has not been able to replicate gender differences from earlier research.

The few studies investigating the role of cognitive load so far find that accommodation is more likely to occur when cognitive load is lower. However, more research is needed to confirm these effects. To our knowledge, the role of cognitive load in long-term accommodation has not been studied to date. A possible way to address this issue is through a longitudinal study with several sessions over a longer period of time. In these sessions, participants would be exposed to a model speaker in two different conditions: One in which the participants solve an easy task and another in which they solve a difficult task and therefore have fewer cognitive resources to attend to the model's speech \citep[see][]{abel_effect_2015}. The hypothesis to be tested is that speech heard while solving an easy task will leave more traces over the long-term than speech heard while solving a difficult task.

From our literature review, several gaps within accommodation research have become evident, which open up the way for new research directions. In particular, the relationship between short- and long-term accommodation, as well as their role in models of language change, remain speculative. First, in long-term accommodation the focus so far has been on migrant communities. Nevertheless, in order to shed light on how accommodation may drive linguistic change, studying the receiving community is as essential as investigating migrating individuals. Second, in both short- and long-term studies the focus has been on adults, who typically show an imperfect acquisition of a new variety. The role of children, who are faster and more complete acquirers of new varieties (and languages), deserves more attention too, and should be better integrated in the change-by-accommodation model. Third, to better understand linguistic accommodation, its underlying mechanism and its ultimate social function, a broader set of languages needs to be studied.

As is evident from the current literature review, research on accommodation so far has mostly focused on well-known Indo-European languages and western communities. Similarly, work on accommodation has typically dealt with phonetics and phonology (especially in short-term studies). More research on different linguistic phenomena and, in particular, direct comparisons between different linguistic levels is crucial to shed light on the mechanisms and constraints of accommodation.

\begin{sloppypar}
Lastly, the striking methodological differences between short- and long-term studies make a direct comparison difficult. In order to study social factors, short-term accommodation research, which typically relies on experimental settings, could benefit from more interactive settings that facilitate spontaneous speech. This is of particular importance because, as mentioned above, the accommodative behavior of a speaker may vary across tasks \citep{pardo_comparison_2018}. Similarly, long-term studies, which so far have mostly relied on sociolinguistic interviews, should use more controlled settings too, to allow for comparability across subjects and  with non-migrant control groups.
\end{sloppypar}

Finally, longitudinal studies will be crucial to offer a more accurate picture of accommodation over longer periods of time. So far, time of exposure has been studied by comparing different individuals. However, given the large inter-speaker variability that pervades published accommodation research, longitudinal studies with data from the same speakers across time are key to understanding accommodation and, in particular, the role of exposure.


%\section*{Abbreviations}
%\section*{Acknowledgements}

\printbibliography[heading=subbibliography,notkeyword=this]


\end{document}
